\documentclass[letterpaper, 10 pt, conference]{ieeeconf}  % Comment this line out if you need a4paper
\pdfoutput=1

%\documentclass[a4paper, 10pt, conference]{ieeeconf}      % Use this line for a4 paper

\IEEEoverridecommandlockouts                              % This command is only needed if 
                                                          % you want to use the \thanks command

\overrideIEEEmargins                                      % Needed to meet printer requirements.

% See the \addtolength command later in the file to balance the column lengths
% on the last page of the document

% The following packages can be found on http:\\www.ctan.org
\usepackage{graphics} % for pdf, bitmapped graphics files
%\usepackage{epsfig} % for postscript graphics files
%\usepackage{mathptmx} % assumes new font selection scheme installed
%\usepackage{times} % assumes new font selection scheme installed
\usepackage{amsmath} % assumes amsmath package installed
\usepackage{amssymb}  % assumes amsmath package installed
\usepackage{graphicx}
% \usepackage{subfig}
\usepackage{hyperref}
\usepackage{subcaption}
\usepackage{graphicx}
% \usepackage{caption}
\usepackage{mathtools}
% \usepackage{enumitem}
\usepackage{dsfont}
\usepackage{float}
\usepackage{makecell}
\usepackage{authblk}
\usepackage{algorithm}
\usepackage{algcompatible}

\pdfminorversion=4


\renewcommand{\baselinestretch}{1.0}

\renewcommand\Authands{, }
\makeatletter
\renewcommand\AB@affilsepx{\hspace{1in} \protect\Affilfont}
\makeatother

\newcommand\blfootnote[1]{%
  \begingroup
  \renewcommand\thefootnote{}\footnote{#1}%
  \addtocounter{footnote}{-1}%
  \endgroup
}

\DeclareMathOperator{\E}{\mathbb{E}}
\newcommand\norm[1]{\left\lVert#1\right\rVert}

\setlength{\belowcaptionskip}{-10pt}

\title{\LARGE \bf
Residual Reinforcement Learning for Robot Control
}

\author{Tobias Johannink$^{*1,3}$, Shikhar Bahl$^{*2}$, Ashvin Nair$^{*2}$, Jianlan Luo$^{1,2}$, Avinash Kumar$^{1}$,\\ Matthias Loskyll$^{1}$, Juan Aparicio Ojea$^{1}$, Eugen Solowjow$^{1}$, Sergey Levine$^2$}

\begin{document}
\maketitle
\blfootnote{$\;^*$ First three authors contributed equally, $\;^1$ Siemens Corporation, \\ $^2$ University of California, Berkeley, $\;^3$ Hamburg University of Technology.}

\thispagestyle{empty}
\pagestyle{empty}


%%%%%%%%%%%%%%%%%%%%%%%%%%%%%%%%%%%%%%%%%%%%%%%%%%%%%%%%%%%%%%%%%%%%%%%%%%%%%%%%
\vspace{-10pt}

\begin{abstract}
Conventional feedback control methods can solve various types of robot control problems 
very efficiently by capturing the structure with explicit models, such as rigid body equations of motion.
However, many control problems in modern manufacturing deal with contacts and friction, which are difficult to capture with first-order physical modeling. 
Hence, applying control design methodologies to these kinds of problems often results in brittle and inaccurate controllers, which have to be manually tuned for deployment.
Reinforcement learning (RL) methods have been demonstrated to be capable of learning continuous robot controllers from interactions with the environment, even for problems that include friction and contacts.
In this paper, we study how we can solve difficult control problems in the real world by decomposing them into a part that is solved efficiently by conventional feedback control methods, and the residual which is solved with RL. 
The final control policy is a superposition of both control signals.
We demonstrate our approach by training an agent to successfully perform a real-world block assembly task involving contacts and unstable objects.
\end{abstract}

%=============================================================================== SECTIONS

\section{Introduction}\label{sec:introduction}

In the previous chapter, we covered a method - residual RL - for incorporating conventional feedback controllers within deep reinforcement learning. 
In this chapter, we will use residual RL specifically for the task of industrial connector insertion.
Many such industrial tasks are on the edge of automation but require a degree of adaptability that is difficult to achieve with conventional robotic automation techniques.
While standard control methods, such as PID controllers, are heavily employed to automate many tasks in the context of positioning, tasks that require significant adaptability or tight visual perception-control loops are often beyond the capabilities of such methods, and therefore are typically performed manually.
Standard control methods can struggle in presence of complex dynamical phenomena that are hard to model analytically, such as complex contacts.
Reinforcement learning (RL) offers a different solution, relying on trial and error learning instead of accurate modeling to construct an effective controller.
RL with expressive function approximation, i.e. deep RL, has further shown to automatically handle high dimensional inputs such as images \citep{mnih2013atari}.

However, deep RL has thus far not seen wide adoption in the automation community due to several practical obstacles. 
Sample efficiency is one obstacle: tasks must be completed without excessive interaction time or wear and tear on the robot. Progress in recent years on developing better RL algorithms has led to significantly better sample efficiency, even in dynamically complicated tasks \citep{haarnoja2018sac, hessel2018rainbow},
but remains a challenge for deploying RL in real-world robotics contexts.
Another major, often underappreciated, obstacle is goal specification: while prior work in RL assumes a reward signal to optimize,
it is often carefully shaped to allow the system to learn \citep{ng1999rewardshaping, popov17stacking, daniel2014activereward}. 
Obtaining such dense reward signals can be a significant challenge, as one must additionally build a perception system that allows computing dense rewards on state representations. Shaping a reward function so that an agent can learn from it is also a manual process that requires considerable manual effort. An ideal RL system would learn from rewards that are natural and easy to specify.
How can we enable robots to autonomously perform complex tasks without significant engineering effort to design perception and reward systems?

\begin{figure}[t]
    \centering
    \begin{subfigure}{0.14\linewidth}
        \center
        USB \vspace{1.2cm}
        
        D-Sub \vspace{1.2cm}
        
        Model-E
    \end{subfigure}
    \begin{subfigure}{0.14\linewidth}
        \center
        \includegraphics[height=1.17cm]{insertion/newfigs/usb.png} \vspace{0.5cm}
        
        \includegraphics[height=1.17cm]{insertion/newfigs/dsub.png} \vspace{0.5cm}
        
        \includegraphics[height=1.17cm]{insertion/newfigs/modele.png} \vspace{0.05cm}
    \end{subfigure}
    \begin{subfigure}{0.69\linewidth}
        \begin{subfigure}[b]{0.69\linewidth}
            \includegraphics[trim={1.7cm 0 0 0},clip,width=0.99\linewidth]{insertion/newfigs/fig_usb.png} 
            \centering
        \end{subfigure}\hspace{2pt}
        \begin{subfigure}[b]{0.29\linewidth}
            \includegraphics[trim={0 0 9cm 0},clip,width=0.99\linewidth]{insertion/figs/robot_view2.jpg}
            \centering
        \end{subfigure}\hfill
    \end{subfigure}

\caption{We train policies directly in the real world to solve connector insertion tasks from raw pixel input and without access to ground-truth state information for reward functions. Left: top-down views of the connectors. Middle: a rollout from a learned policy that successfully completes the insertion task for each connector is shown. Right: a full view of the robot setup. Videos of the results are available at \href{https://industrial-insertion-rl.github.io/}{industrial-insertion-rl.github.io} }
    \label{fig:usb_photo_demo2}
\end{figure}

We first consider an end-to-end approach that learns a policy from images, where the images serve as both the state representation and the goal specification. Using goal images is not fully general, but can successfully represent tasks when the task is to reach a final desired state \citep{nair2018rig}.
Specifying goals via goal images is convenient, and makes it possible to specify goals with minimal manual effort. Using images as the state representation also allows a robot to learn behaviors that utilize direct visual feedback, which provides some robustness to sensor and actuator noise.

Secondly, we consider learning from simple and sparse reward signals. Sparse rewards can often be obtained conveniently, for instance from human-provided labels or simple instrumentation. In many electronic assembly tasks, which we consider here, we can directly detect whether the electronics are functional, and use that signal as a reward. Learning from sparse rewards poses a challenge, as exploration with sparse reward signals is difficult, but by using sufficient prior information about the task, one can overcome this challenge. To handle this challenge, we extend the residual RL approach~\cite{johannink18residualrl, silver18residualpolicylearning}, which learns a parametric policy on top of a fixed, hand-specified controller, to the setting of vision-based manipulation.

% \begin{figure}[t]
%     \centering
%         \begin{subfigure}[b]{0.70\linewidth}
%             \includegraphics[width=0.99\linewidth]{insertion/newfigs/fig_ports.png} 
%             \centering
%         \end{subfigure}%\hfill
    
%     \caption{A close-up view of the three connector insertion tasks shows the contacts and tight tolerances the agent must navigate to solve these tasks. These tasks require sub-millimeter precision without visual feedback. }
%     \label{fig:usb_photo_demo}
% \end{figure}

In our experiments, we show that we can successfully complete real-world tight tolerance assembly tasks, such as inserting USB connectors, using RL from images with reward signals that are convenient for users to specify.
We can learn from only a sparse reward based on the electrical connection for a USB adapter plug, and we demonstrate learning insertion skills with rewards based only on goal images.
These reward signals require no extra engineering and are easy to specify for many tasks. 
Beyond showing the feasibility of RL for solving these tasks, we evaluate multiple RL algorithms across three tasks and study their robustness to imprecise positioning and noise.

% \section{Related Work}\label{sec:relatedwork}

Learning has been applied previously in a variety of robotics contexts. Different forms of learning have enabled autonomous driving \cite{pomerleau1989alvinn}, biped locomotion \cite{nakanishi2004bipedlfd}, block stacking \cite{deisenroth2011stacking}, grasping \cite{pinto2015supersizing}, and navigation \cite{giusti15trails, pathak2018zeroshot}. Among these methods, many involve reinforcement learning, where an agent learns to perform a task by maximizing a reward signal. Reinforcement learning algorithms have been developed and applied to teach robots to perform tasks such as balancing a robot \cite{deisenroth2011pilco}, playing ping-pong \cite{peters2010reps} and baseball \cite{peters2008baseball}.
The use of large function approximators, such as neural networks, in RL has further broadened the generality of RL \cite{mnih2013atari}. Such techniques, called ``deep'' RL, have further allowed robots to be trained directly in the real world to perform fine-grained manipulation tasks from vision \cite{levine2016gps}, open doors \cite{gu2016naf}, play hockey \cite{chebotar2017pilqr}, stack Lego blocks \cite{zhang2019solar}, use dexterous hands \cite{zhu2019hands}, and grasp objects \cite{kalashnikov2018qtopt}. In this work we further explore solving real-world robotics tasks using RL.

Many RL algorithms introduce prior information about the specific task to be solved. One common method is reward shaping \cite{ng1999rewardshaping}, but reward shaping can become arbitrarily difficult as the complexity of the task increases. Other methods incorporate a trajectory planner \cite{thomas2018cad} but for complex assembly tasks, trajectory planners require a host of information about objects and geometries which can be difficult to provide.

Another body of work on incorporating prior information studies using  demonstrations either to initialize a policy \cite{peters2008baseball, kober2008mp}, infer reward functions using inverse reinforcement learning \cite{finn16guidedcostlearning, ziebart2008maxent} or to improve the policy throughout the learning procedure \cite{hester17dqfd, nair2018demonstrations, rajeswaran2018dextrous}. These methods require multiple demonstrations, which can be difficult to collect, especially for assembly tasks, although learning a reward function by classifying goal states \cite{singh2019raq} may partially alleviate this issue. More recently, manually specifying a policy and learning the residual task has been proposed \cite{johannink18residualrl, silver18residualpolicylearning}. In this work we evaluate both residual RL and combining RL with learning from demonstrations.

Previous work has also tackled high precision assembly tasks, especially insertion-type tasks. One line of work focuses on obtaining high dimensional observations, including geometry, forces, joint positions and velocities \cite{li2014usbgelsight, tamar2017hindsightplan, inoue2017deeprlassembly, luo19variableimpedance}, but this information is not easily procured, increasing complexity of the experiments and the supervision required. Other work relies on external trajectory planning or very high precision control \cite{inoue2017deeprlassembly, tamar2017hindsightplan}, but this can be brittle to error in other components of the system, such as perception. We show how our method not only solves insertion tasks with much less information about the environment, but also does so under noisy conditions. %% related work moved after results 
\section{Preliminaries}\label{sec:background}
In this section, we set up our problem and summarize the foundations of classical control and reinforcement learning that we build on in our approach.

\subsection{Problem Statement - System Theoretic Interpretation}
%
The class of control problems that we are dealing with in this paper can be viewed from a dynamical systems point of view as follows.
Consider a dynamical system that consists of a fully actuated robot and underactuated objects in the robot's environment.
The robot and the objects in its environment are described by their states $s_\text{m}$ and $s_\text{o}$, respectively. 
The robot can be controlled through the control input $u$ while the objects cannot be directly controlled.
However, the robot's states are coupled with the objects' states so that indirect control of $s_\text{o}$ is possible through $u$.
This is for example the case if the agent has large inertia and is interacting with small parts as is common in manufacturing.
The states of agent and objects can either be fully observable or they can be estimated from measurements.

The time-discrete equations of motion of the overall dynamical system comprise the robot and objects and can be stated as
%
\begin{equation}
\label{eq:eom}
    s_{t+1}=\begin{bmatrix}s_{\text{m},t+1} \\ s_{\text{o}, t+1}\end{bmatrix}
    =\begin{bmatrix}A(s_{\text{m},t}) & 0 \\ B(s_{\text{m},t}, s_{\text{o},t}) & C(s_{\text{o},t})\end{bmatrix}\begin{bmatrix}s_{\text{m},t} \\ s_{\text{o},t}\end{bmatrix}  + D\begin{bmatrix}u_t \\ 0\end{bmatrix},
\end{equation}
%
where the states can also be subject to algebraic constraints, which we do no state explicitly here.

The type of control objectives that we are interested in can be summarized as controlling the agent in order to manipulate the objects while also fulfilling a geometric objective such as trajectory following.
It is difficult to solve the control problem directly with conventional feedback control approaches, which compute the difference between a desired and a measured state variable.
In order to achieve best system performance feedback control methods require well understood and modeled state transition dynamics. 
Finding the optimal control parameters can be difficult or even impossible if the system dynamics are not fully known.

In \eqref{eq:eom} the state transition matrices although $A(s_\text{m})$ and $C(s_\text{o})$ are usually known to a certain extent, because they represent rigid body dynamics, the coupling matrix $B(s_\text{m}, s_\text{o})$ is usually not known. 
Physical interactions such as contacts and friction forces are the dominant effects that $B(s_\text{m}, s_\text{o})$  needs to capture, which also applies to algebraic constraints, which are functions of $s_\text{m}$ and $s_\text{o}$ as well. 
Hence, conventional feedback control synthesis for determining $u$ to control $s_\text{o}$ is very difficult, and requires trial and error in practice.
Another difficulty for directly designing feedback controllers is due to the fact that, for many control objectives, the states $s_\text{o}$ need to fulfill conditions that cannot be expressed as deviations (errors) from desired states. 
This is often the case when we only know the final goal rather than a full trajectory.

Instead of directly designing a feedback control system, we can instead specify the goal via a reward function. These reward functions can depend on both $s_\text{m}$ and $s_\text{o}$, where the terms that depend on $s_\text{m}$ are position related objectives.
% \subsection{Interpretation as a Reinforcement Learning Problem}
% In reinforcement learning, we consider the standard Markov decision process framework for picking optimal actions to maximize rewards over discrete timesteps in an environment $E$. At every timestep $t$, an agent is in a state $s_t$, takes an action $u_t$, receives a reward $r_t$, and $E$ evolves to state $s_{t+1}$. In reinforcement learning, the agent must learn a policy $u_t = \pi(s_t)$ to maximize expected returns.  We denote the return by $R_t = \sum_{i=t}^T \gamma^{(i - t)} r_i$, where $T$ is the horizon that the agent optimizes over and $\gamma$ is a discount factor for future rewards. The agent's objective is to maximize expected return from the start distribution $J = \E_{r_i, s_i \sim E, a_i \sim \pi}[R_0]$. 
Reinforcement learning can be used to maximize the reward function in a model-free way. In reinforcement learning, we simply attempt to maximize expected return. Unlike the previous section, RL does not attempt to model the unknown coupled dynamics of the agent and the object. Instead, it finds actions that maximizes rewards, without making any assumptions about the system dynamics. The final objective is to learn a policy $u_t = \pi(s_t)$ to maximize expected returns $R_t = \sum_{i=t}^T \gamma^{(i - t)} r_i$.
\section{Method}\label{sec:method}

Based on the analysis in Sec. \ref{sec:background}, we introduce a control system that consists of two parts. The first part is based on conventional feedback control theory and maximize all reward terms that are functions of $s_\text{m}$. An RL method is superposed and maximizes the reward terms that are functions of $s_\text{o}$.

\subsection{Residual Reinforcement Learning}
In most robotics tasks, we consider rewards of the form:
%
\begin{equation}\label{eq:reward}
    r_t = f(s_\text{m}) + g(s_\text{o}).
\end{equation}
%
The term $f(s_\text{m})$ is assumed to be a function, which represents a geometric relationship of robot states, such as a Euclidean distance or a desired trajectory.
%% AVN is this a necessary assumption?
The second term of the sum $g(s_\text{o})$ can be a general class of functions. Concretely, in our model assembly task, $f(s_m)$ is the reward for moving the robot gripper between the standing blocks, while $g(s_o)$ is the reward for keeping the standing blocks upright and in their original positions.

The key insight of residual RL is that in many tasks, $f(s_\text{m})$ can be easily optimized a priori of any environment interaction by conventional controllers, while $g(s_\text{o})$ may be easier to learn with RL which can learn fine-grained hand-engineered feedback controllers even with friction and contacts.
To take advantage of the efficiency of conventional controllers but also the flexbility of RL, we choose:
%
\begin{equation}\label{eq:ctrl_seq}
    u = \pi_H(s_\text{m}) + \pi_\theta(s_\text{m}, s_\text{o})
\end{equation}
%
as the control action, where $\pi_H(s_\text{m})$ is the human-designed controller and $\pi_\theta(s_\text{m}, s_\text{o})$ is a learned policy parametrized by $\theta$ and optimized by an RL algorithm to maximize expected returns on the task.

Inserting \eqref{eq:ctrl_seq} into \eqref{eq:eom} one can see that a properly designed feedback control law for $\pi_H(s_\text{m})$ is able to provide exponentially stable error dynamics of $s_\text{m}$ if the learned controller $\pi_\theta$ is neglected and the sub statespace is stabilizable.
This is equivalent to maximizing \eqref{eq:reward} for the case $f$ represents errors between actual and desired states.

The residual controller $\pi_\theta(s_\text{m}, s_\text{o})$ can now be used to maximize the reward term $g(s_\text{o})$ in \eqref{eq:reward}.
Since the control sequence \eqref{eq:ctrl_seq} enters \eqref{eq:eom} through the dynamics of $s_\text{m}$ and $s_\text{m}$ is in fact the control input to the dynamics of $s_\text{o}$, we cannot simply use the a-priori hand-engineered feedback controller to achieve zero error of $s_\text{m}$ and independently achieve the control objective on $s_\text{o}$.
Through the coupling of states we need to perform an overall optimization of \eqref{eq:ctrl_seq}, whereby the hand-engineered feedback controller provides internal structures and eases the optimization related to the reward term $f(s_\text{m})$.

\setlength{\textfloatsep}{0.09cm}
\begin{algorithm}
   	\caption{Residual reinforcement learning}
   	\label{alg:residualrl}
   	\begin{algorithmic}[1]
    \REQUIRE policy $\pi_\theta$, hand-engineered controller $\pi_\text{H}$.
    \FOR{$n=0,...,N-1$ episodes}
        \STATE Initialize random process $\mathcal{N}$ for exploration
        \STATE Sample initial state $s_0 \sim E$.
        \FOR{$t=0,...,H -1$ steps}
            \STATE Get policy action $u_t = \pi_\theta(s_t) + \mathcal{N}_t$.
            \STATE Get action to execute $u'_t = u_t + \pi_\text{H}(s_t)$.
            \STATE Get next state $s_{t+1} \sim p(\cdot \mid s_t, u'_t)$.
            \STATE Store $(s_t, u_t, s_{t+1})$ into replay buffer $\mathcal R$.
            \STATE Sample set of transitions $(s, u, s') \sim \mathcal R$.
            \STATE Optimize $\theta$ using RL with sampled transitions.
        \ENDFOR
    \ENDFOR
   	\end{algorithmic}
\end{algorithm}
\setlength{\floatsep}{0.09cm}
\subsection{Method Summary}
Our method is summarized in Algorithm \ref{alg:residualrl}. The key idea is to combine the flexibility of RL with the efficiency of conventional controllers by additively combining a learnable parametrized policy with a fixed hand-engineered controller.

\begin{figure*}[t]
    \vspace{6pt}
    \centering
    \includegraphics[height=1.1in]{residualrl/figs/sim_learning_curve3.pdf}
    \includegraphics[height=1.1in]{residualrl/figs/sawyer_env.png} 
    \hspace{1cm}
    \includegraphics[height=1.1in]{residualrl/figs/real_world_env.png} 
    \vspace{0.1cm}
    \includegraphics[height=1.1in]{residualrl/figs/real_residualrl_comparison.pdf}
    \caption{Block assembly task in simulation (left) and  real-world (right). The task is to insert a block between the two blocks on the table without moving the blocks or tipping them over. In the learning curves, we compare our method with RL without any hand-engineered controller\protect\footnotemark. In both simulation and real-world experiments, we see that residual RL learns faster than RL alone, while achieving better performance than the hand-engineered controller. }%
    \label{fig:fig2}
\end{figure*}

As our underlying RL algorithm, we use a variant of twin delayed deep deterministic policy gradients (TD3)
as described in~\cite{fujimoto2018td3}. TD3 is a value-based RL algorithm for continuous control based off of the deep deterministic policy gradient (DDPG) algorithm \cite{lillicrap2015continuous}. We have found that TD3 is stable, sample-efficient, and requires little manual tuning compared to DDPG.
We used the publicly available \href{https://github.com/vitchyr/rlkit}{\texttt{rlkit}} implementation of TD3 \cite{pong2018tdm}. Our method is independent of the choice of RL algorithm, and we could apply residual RL to any other RL algorithm.
\section{Experimental Setup}\label{sec:experiments}

We evaluate our method on the task shown in Fig. \ref{fig:fig2}, both in simulation and in the real world. This section introduces the details of the experimental setup and provides an overview of the experiments.

\subsection{Simulated Environment}
We use MuJoCo~\cite{todorov12mujoco}, a full-featured simulator for model-based optimization considering body contacts, to evaluate our method in simulation.
This environment consists of a simulated Sawyer robot arm with seven degrees of freedom and a parallel gripper. 
We command the robot with a Cartesian-space position controller.
%% Cut for space
% Two blocks each with 3-DOF and one angled corner on the top are placed on a defined platform on the table in front of the robot. 
% To allow block insertion between the standing blocks, a sufficiently large gap is defined (this gap represents the goal position). 
% Both standing blocks can slide in $x$- and $y$-direction and topple around the $y$-axis.
% The agent receives the end effector position, the end effector forces and torques in relation to the $x$-, $y$- and $z$-axes, all block positions, and the goal position as the observation. In the robot's initial position one block for the insertion process is already picked up by the gripper claws and the gripper is located above the blocks on the tabletop. 

% We use a reward function 
%     \begin{align}
%         r_t = - \|x_g - x_t \|_2 - \lambda (\|\theta_{l} \|_1 + \|\theta_{r} \|_1)
%     \end{align}
% where $x_t$ is the current block position, $x_g$ is the goal position, $\theta_{l}$, $\theta_{r}$ are the angles with respect to the table (in the y-axis) of the left and right blocks, respectively. $\lambda$ is a hyperparameter. 


\subsection{Real-World Environment}
The real-world environment is largely the same as the simulated environment, except for the controller, rewards, and observations.
We command the robot with a compliant joint-space impedance controller we have developed to be smooth and tolerant of contacts.
The positioning of the block being inserted is similar to the simulation but the observation is estimated from a camera-based tracking system as we do not have access to ground truth position information.
%% Cut for space
% Due to the blocks' slight weight and their capability of sliding in the plane ($x$, $y$), the Sawyer is not able to measure contact forces regarding these axes.
% Therefore, we only add the end effector forces in $z$-direction to the observation space instead of observing the end effector forces and torques regarding to the $x$-, $y$- and $z$-axes.
% The reward function was slightly different, being defined as: 
%     \begin{align}
%     \begin{split}
%     r_t = - \|x_g - x_t \|_2 - \lambda (\|\theta_{l} \|_1 + \|\theta_{r} \|_1) \\ - \mu \|X_g - X_t \|_2 - \beta (\|\phi_{l} \|_1 + \|\phi_{r} \|_1)
%     \end{split}
%     \end{align}
% where $x_t$ is the current end effector position, $x_g$ is the goal position, $X_t$ describes the current position of both standing blocks, $X_g$ their desired positions, $\theta_{l}$, $\theta_{r}$ are the angles of the current orientation with respect to the table (in the y-axis) and $\phi_{l}$ and $\phi_{r}$ are the angles of the current orientation with respect to the z-axis of the left and the right block respectively $\lambda$, $\mu$, and $\beta$ are hyperparameters. 

%
% \subsection{Training Details}

\footnotetext{In all simulation plots, we use 10 random seeds and report a $95\%$ confidence interval for the mean.}

\subsection{Overview of Experiments}
%%SL.12.01: This feels like it belongs in the next section
%
In our experiments we evaluate the following research questions:
\begin{enumerate}
    \item Does incorporating a hand-designed controller improve the performance and sample-efficiency of RL algorithms, while still being able to recover from an imperfect hand-designed controller?
    \item Can our method allow robots to be more tolerant of variation in the environment?
    \item Can our method successfully control noisy systems, compared to classical control methods?
\end{enumerate}

\section{Experiments}

\subsection{Sample Efficiency of Residual RL}
In this section, we compare our residual RL method with the human controller alone and RL alone. The following methods are compared:
\begin{enumerate}
    \item Only RL: using the same underlying RL algorithm as our method but without adding a hand-engineered policy
    \item Residual RL: our method which trains a superposition of the hand-engineered controller and a neural network policy, with RL
\end{enumerate}

\subsection{Effect of Environment Variation}
In automation, environments can be subject to noise and solving manufacturing tasks become more difficult as variability in the environment increases. 
It is difficult to manually design feedback controllers that are robust to environment variation, as it might require significant human expertise and tuning. In this experiment, we vary the initial orientation of the blocks during each episode and demonstrate that residual RL can still solve the task. We compare its performance to that of the hand-engineered controller.

To introduce variability in simulation, on every reset we sampled the rotation of each block independently from a uniform distribution $U[-r, r], r \in \{0, 0.05, 0.1, 0.15, 0.2, 0.25, 0.3\}$.

Similarly, in the real world experiments, on every reset we randomly rotated each block to one of three orientations: straight, tilted clockwise, or tilted counterclockwise (tilt was $\pm$  $20^{\circ}$ from original position).

\subsection{Recovering from Control Noise}

Due to a host of issues, such as defective hardware or poorly tuned controller parameters, feedback controllers might have induced noise. Conventional feedback control policies are determined a priori and do not adapt their behavior from data.
However, RL methods are known for their ability to cope with shifting noise distributions and are capable of recovering from such issues.

\begin{figure}[ht!]
    \vspace{6pt}
    \centering
    \begin{subfigure}[b]{0.38\linewidth}
    \renewcommand{\arraystretch}{1.5}
    \footnotesize
        \begin{tabular}{ | c || c | c |}
            \hline
            Misaligned? & No & Yes  \\ \hline
            Residual RL & 20/20 & 15/20  \\ \hline
            Hand-engineered  & 20/20 & 2/20  \\ \hline
        \end{tabular}
        \vspace{.8cm} \\
        \centering
        \normalsize{(a)}
    \end{subfigure}
    \begin{subfigure}[b]{0.3\linewidth}
        \includegraphics[width=0.99\linewidth]{residualrl/figs/real_world_envvar_success.pdf} \\
        \centering
        (b)
    \end{subfigure}
    \begin{subfigure}[b]{0.3\linewidth}
        \includegraphics[width=0.99\linewidth]{residualrl/figs/real_world_control_bias.pdf} \\
        \centering
        (c)
    \end{subfigure}
    \vspace{0cm}
    \caption{Outcome of our residual RL method in different experiments during the block assembly task in the real-world. Success rate is recorded manually by human judgment of whether the blocks stayed upright and ended in the correct position. Plot (a) compares the insertion success of residual RL and hand-designed controller depending on the block orientation during run time. Plot (b) shows the success rate of the insertion process during training, where on every reset the blocks are randomly rotated: straight, tilted clockwise, or tilted counterclockwise ($\pm$  $20^{\circ}$) and plot (c) shows the increasing success rate of our method for biased controllers as well even as control bias increases.}%
    \label{fig:environment_variation}
\end{figure}

In this experiment, we introduce a control noise, including biased control noise, and demonstrate that residual RL can still successfully solve the task, while a hand-engineered controller cannot. The control noise follows a normal distribution and is added to the control output of the system at every step:
\begin{align}
    u'_t = u_t + \mathcal{N}(\mu, \sigma^2)
\end{align}

To test tolerance to control noise, we set $\mu = 0$ and vary $\sigma \in [0.01, 0.1]$. In theory, RL could adapt to a noisy controller by learning more robust solutions to the task which are less sensitive to perturbations.

Furthermore, to test tolerance to a biased controller, we set $\sigma = 0.05$ and vary $\mu \in [0, 0.2]$. To optimize the task reward, RL can learn to simply counteract the bias. 

\subsection{Sim-to-Real with Residual RL}

As an alternative to analytic solutions of real-world control problems, we can often instead model the forward dynamics of the problem (ie. a simulator). With access to such a model, we can first find a solution to the problem with our possibly inaccurate model, and then use residual RL to find a realistic control solution in the real world.

In this experiment, we attempt the block insertion task with the side blocks fixed in place. The hand-engineered policy $\pi_\text{H}$ in this case comes from training a parametric policy in simulation of the same scenario (with deep RL). We then use this policy as initialization for residual RL in the real world.
\begin{figure}
\resizebox{\textwidth}{!}{
\begin{minipage}[t]{.5\textwidth}
\small
    \centering
        \begin{tabular}{|l|l|l|l|}
        \hline
        \multicolumn{2}{|l|}{\multirow{2}{*}{D-Sub Connector}} & \multicolumn{2}{c|}{Goal}                                 \\ \cline{3-4} 
        \multicolumn{2}{|l|}{}                     & \multicolumn{1}{c|}{Perfect} & \multicolumn{1}{c|}{Noisy} \\ \hline \hline
        \multirow{3}{*}{Pure RL} & Dense & 16\% & 0\% \\ 
         & Images, SAC & 0\% & 0\% \\
          & Images, TD3 & 12\% & 12\% \\ \hline
        \multirow{1}{*}{RL + LfD} & Images & 52\% & 52\% \\  \hline
        \multirow{3}{*}{Residual RL} & Dense & \textbf{100\%} & 60\% \\ 
         & Images, SAC & \textbf{100\%} & \textbf{64\%} \\
          & Images, TD3 & 52\% &  52\% \\ \hline
        \multirow{1}{*}{Human} & P-Controller & \textbf{100\%} & 44\% \\   \hline
        \end{tabular}
\end{minipage}
\begin{minipage}[t]{.5\textwidth}
\renewcommand{\arraystretch}{1}
\small
\begin{tabular}{|l|l|l|l|}
        \hline
        \multicolumn{2}{|l|}{\multirow{2}{*}{Model-E Connector}} & \multicolumn{2}{c|}{Goal}                                 \\ \cline{3-4} 
        \multicolumn{2}{|l|}{}                     & \multicolumn{1}{c|}{Perfect} & \multicolumn{1}{c|}{Noisy} \\ \hline \hline
        \multirow{3}{*}{Pure RL} & Dense & 0\% & 0\% \\ 
         & Images, SAC & 0\% & 0\% \\ 
         & Images, TD3 & 0\% & 0\% \\ \hline
        \multirow{1}{*}{RL + LfD} & Images & 20\% & 20\% \\  \hline
        \multirow{3}{*}{Residual RL} & Dense & \textbf{100\%} & \textbf{76\%} \\ 
         & Images, SAC &\textbf{100\%} & \textbf{76\%} \\ 
           & Images, TD3 & 0\% & 0\% \\ \hline
        \multirow{1}{*}{Human} & P-Controller & 52\% & 24\% \\   \hline
        \end{tabular}
\end{minipage}}
\captionof{table}{We report average success out of 25 policy executions after training is finished for each method. For noisy goals, noise is added in form of $\pm 1\,\mathrm{mm}$ perturbations of the goal location. Residual RL, particularly with SAC, tends to be the best performing method across all three connectors. For the \text{Model-E}~connector, only residual RL solves the task in the given amount of training time.}
    \label{fig:tables}
    
\end{figure}


        
       

\section{Results}\label{sec:results}

We analyze the performance of policies learned with residual RL, as well as other methods, based on their ability to achieve the task goal, as well as the distance of the final object location to the goal pose over the course of training. To study the robustness of the learned policies, we also evaluate them in conditions where the goal connector position is perturbed, in order to understand the tolerance of RL policies to imprecise object placement.

\subsection{Vision-based Learning}
The results of the vision-based experiment are shown in Fig.~\ref{vision_based_distance_all}.
Our experiments show that a successful and consistent vision-based insertion policy can be learned from relatively few samples using residual RL. 

This result suggests that goal-specification through images is a practical way to solve these types of industrial tasks. Although image-based rewards are often very sparse and hard to learn from, in this case the distance between images corresponds to a relatively dense reward signal which is sufficient to distinguish the different stages of the insertion process.

Interestingly, during training with standard RL, the policy would sometimes learn to ``hack'' the reward signal by moving down in the image in front of or behind the socket. In contrast, the stabilizing human-engineered controller in residual RL provides sufficient horizontal control to prevent this. The initial controller also scaffolds the learning process, by providing a very strong initialization that requires the reinforcement learning algorithm to only learn the final phase of the insertion. This produces substantially better performance in conjunction with vision-based rewards.

\vspace{.3cm}
\begin{table}[ht!]
    \centering
\begin{minipage}[t]{.47\textwidth}
    \renewcommand{\arraystretch}{1}
    \footnotesize
    \begin{tabular}{|l|l|l|l|}
    \hline
    \multicolumn{2}{|l|}{\multirow{2}{*}{USB Connector}} & \multicolumn{2}{c|}{Goal}                                 \\ \cline{3-4} 
    \multicolumn{2}{|l|}{}                     & \multicolumn{1}{c|}{Perfect} & \multicolumn{1}{c|}{Noisy} \\ \hline \hline
    \multirow{5}{*}{Pure RL}   
     & Dense & 28\% & 20\% \\ 
     & Sparse, \;SAC & 16\% & 8\% \\ 
     & Sparse, \;TD3 & 44\% & 28\% \\ 
      & Images, SAC & 36\% & 32\% \\ 
       & Images, TD3 & 28\% & 28\% \\ \hline
    \multirow{2}{*}{RL + LfD} & Sparse &\textbf{100\%} & 32\% \\ 
     & Images & 88\% & 60\% \\ \hline
    \multirow{5}{*}{Residual RL} & Dense &\textbf{100\%} & \textbf{84\%} \\ 
     & Sparse,\: SAC & 88\% &  \textbf{84\%}\\ 
     & Sparse,\: TD3 &\textbf{100\%} & 36\% \\
     & Images, SAC & \textbf{100\%} & 80\% \\
      & Images, TD3 &  0\%& 0\%\\
      \hline
    \multirow{1}{*}{Human} & P-Controller & \textbf{100\%} & 60\% \\   \hline
    \end{tabular}
    %\captionsetup{justification=justified,format=plain}
    %\captionof{table}{Average success rate on the USB insertion task. Residual RL and RL + LfD solve the task consistently. Moreover, residual RL stays robust under $\pm1$\text{mm} noise. }
    \label{tab:goal_pertubation_USB}
    \label{fig:TableUSB}
\end{minipage}
\hfill
\begin{minipage}[t]{.49\textwidth}
\centering
    \includegraphics[width=1\linewidth]{insertion/newfigs/Sparse_success_all.pdf}
    \captionsetup{justification=justified,format=plain}
    \captionof{figure}{Learning curves for solving the USB insertion task with a sparse reward. Final distance to goal is shown; lower is better. Residual RL and RL with learning from demonstrations both solve the task relatively quickly, while RL alone takes about twice as long to solve the task at the same performance. }
    \label{fig:SparseRewardsAll}
\end{minipage}
\caption{Learning curves for solving the USB insertion task with a sparse reward. Final distance to goal is shown; lower is better. Residual RL and RL with learning from demonstrations both solve the task relatively quickly, while RL alone takes about twice as long to solve the task at the same performance. }
\end{table}

\subsection{Learning From Sparse Rewards}

In this experiment, we compare these methods on the USB insertion task with sparse rewards. The results are reported in Fig.~\ref{fig:SparseRewardsAll}. All methods are able to achieve very high success rates in the sparse setting. 
This result shows that we can learn precise industrial insertion tasks in sparse-reward settings, which can often be obtained much more easily than a dense, shaped reward. 
In fact, prior work has found that the final policy for sparse rewards can outperform the final policy for dense rewards as it does not suffer from a misspecified objective \citep{andrychowicz2017her}.

\subsection{Perfect State Information}
The results of the experiment with perfect state information and dense rewards is shown in Fig.~\ref{fig:state_based_distance_all}. In this case, residual RL still outperforms standard RL, though the better-shaped reward enables standard RL to make more initial progress than with the other reward signals.
However, the hand-designed shaped reward function makes it harder for the policy to actually perform the full insertion, potentially because the more complex reward landscape provides other competing goals to the policy. The final performance with sparse rewards on the USB insertion task is substantially better.


\begin{figure}[tbp]
    \centering
    \resizebox{1.02\linewidth}{!}{
    \begin{subfigure}[b]{0.32\linewidth}
        \includegraphics[width=0.99\linewidth]{insertion/newfigs/USB_state_distance.pdf}
    \end{subfigure}
    \begin{subfigure}[b]{0.32\linewidth}
        \includegraphics[width=0.99\linewidth]{insertion/newfigs/DSUB_states_distance.pdf}
    \end{subfigure}
    \begin{subfigure}[b]{0.32\linewidth}
        \includegraphics[width=0.99\linewidth]{insertion/newfigs/Waterproof_state_distance.pdf}
    \end{subfigure}
    }
    \caption{Plots of the final mean distance to the goal during the state-based training. Final distances greater than $0.01\,\mathrm{m}$ indicate unsuccessful insertions. Here, the residual RL approach performs noticeably better than pure RL and is often able to solve the task during the exploration in the early stages of the training.}
    \label{fig:state_based_distance_all}
    % \vspace{-0.5cm}
\end{figure}

\subsection{Robustness}

In the previous set of experiments, the goal locations were known exactly.
In this case, the hand-engineered controller
performs well. However, once
noise is added to the goal location, the deterministic P-controller struggles. 
To test robustness, a goal perturbation is created artificially, and the controllers are tested under this condition. 
All results of our robustness evaluations are listed in Fig.~\ref{fig:tables} and Fig.~\ref{fig:TableUSB}.  
In the presence of a $\pm 1\mathrm{mm}$ perturbation, the
residual RL 
controller succeeds more often on the USB and D-Sub tasks, and significantly more often on the Model-E task.
Unlike the hand-engineered controller, residual RL consistently solved this task and overcame goal perturbations in $16/25$ trials. 
The agent demonstrably learns small but consistent corrective
feedback behaviors in order to move in
the right direction during the descent motion, a behavior that
is very difficult to specify manually.
This behavior illustrates the strength of
residual RL. Since the human controller already specifies the general
trajectory of the optimal policy, environment samples are only
required to learn this corrective feedback behavior.

\iffalse
\begin{figure}[thb]
    \centering
    \begin{subfigure}[b]{0.20\linewidth}
        \includegraphics[width=0.99iew screeen\linewidth]{insertion/robot_view_screenshots/S11.png}\\
        \centering
    \end{subfigure} \hfill
    \begin{subfigure}[b]{0.20\linewidth}
        \includegraphics[width=0.99\linewidth]{insertion/robot_view_screenshots/S21.png}\\
        \centering
    \end{subfigure}   \hfill
    \begin{subfigure}[b]{0.20\linewidth}
        \includegraphics[width=0.99\linewidth]{insertion/robot_view_screenshots/S31.png}\\
        \centering
    \end{subfigure}   \hfill
    \begin{subfigure}[b]{0.20\linewidth}
        \includegraphics[width=0.99\linewidth]{insertion/robot_view_screenshots/S41.png}\\
        \centering
    \end{subfigure}      
    \\
    \vspace{10pt}
    \begin{subfigure}[b]{0.20\linewidth}
        \includegraphics[width=0.99\linewidth]{insertion/robot_view_screenshots/S12.png}\\
        \centering
    \end{subfigure} \hfill
    \begin{subfigure}[b]{0.20\linewidth}
        \includegraphics[width=0.99\linewidth]{insertion/robot_view_screenshots/S22.png}\\
        \centering
    \end{subfigure}     \hfill
    \begin{subfigure}[b]{0.20\linewidth}
        \includegraphics[width=0.99\linewidth]{insertion/robot_view_screenshots/S32.png}\\
        \centering
    \end{subfigure}  \hfill
    \begin{subfigure}[b]{0.20\linewidth}
        \includegraphics[width=0.99\linewidth]{insertion/robot_view_screenshots/S42.png}\\
        \centering
    \end{subfigure}    
    \\
    \vspace{10pt}
    \begin{subfigure}[b]{0.20\linewidth}
        \includegraphics[width=0.99\linewidth]{insertion/robot_view_screenshots/DSUB_s1.png}\\
        \centering
    \end{subfigure} \hfill
    \begin{subfigure}[b]{0.20\linewidth}
        \includegraphics[width=0.99\linewidth]{insertion/robot_view_screenshots/DSUB_s2.png}\\
        \centering
    \end{subfigure}     \hfill
    \begin{subfigure}[b]{0.20\linewidth}
        \includegraphics[width=0.99\linewidth]{insertion/robot_view_screenshots/DSUB_s4.png}\\
        \centering
    \end{subfigure}  \hfill
    \begin{subfigure}[b]{0.20\linewidth}
        \includegraphics[width=0.99\linewidth]{insertion/robot_view_screenshots/DSUB_s5.png}\\
        \centering
    \end{subfigure}    
    \\
    \vspace{10pt}
    \begin{subfigure}[b]{0.20\linewidth}
        \includegraphics[width=0.99\linewidth]{insertion/robot_view_screenshots/dsub_target_image1.png}\\
        \centering
    \end{subfigure} \hfill
    \begin{subfigure}[b]{0.20\linewidth}
        \includegraphics[width=0.99\linewidth]{insertion/robot_view_screenshots/dsub_target_image2.png}\\
        \centering
    \end{subfigure}     \hfill
    \begin{subfigure}[b]{0.20\linewidth}
        \includegraphics[width=0.99\linewidth]{insertion/robot_view_screenshots/dsub_target_image3.png}\\
        \centering
    \end{subfigure}  \hfill
    \begin{subfigure}[b]{0.20\linewidth}
        \includegraphics[width=0.99\linewidth]{insertion/robot_view_screenshots/dsub_target_image3.png}\\
        \centering
    \end{subfigure}    
    \\
    \vspace{10pt}
    \begin{subfigure}[b]{0.20\linewidth}
        \includegraphics[width=0.99\linewidth]{insertion/robot_view_screenshots/waterproof_s2.png}\\
        \centering
    \end{subfigure} \hfill
    \begin{subfigure}[b]{0.20\linewidth}
        \includegraphics[width=0.99\linewidth]{insertion/robot_view_screenshots/waterproof_s3_.png}\\
        \centering
    \end{subfigure}     \hfill
    \begin{subfigure}[b]{0.20\linewidth}
        \includegraphics[width=0.99\linewidth]{insertion/robot_view_screenshots/waterproof_s4.png}\\
        \centering
    \end{subfigure}  \hfill
    \begin{subfigure}[b]{0.20\linewidth}
        \includegraphics[width=0.99\linewidth]{insertion/robot_view_screenshots/waterproof_s7.png}\\
        \centering
    \end{subfigure}    
    \\
    \vspace{10pt}
    \begin{subfigure}[b]{0.20\linewidth}
        \includegraphics[width=0.99\linewidth]{insertion/robot_view_screenshots/water_target_image1.png}\\
        \centering
    \end{subfigure} \hfill
    \begin{subfigure}[b]{0.20\linewidth}
        \includegraphics[width=0.99\linewidth]{insertion/robot_view_screenshots/water_target_image2.png}\\
        \centering
    \end{subfigure}     \hfill
    \begin{subfigure}[b]{0.20\linewidth}
        \includegraphics[width=0.99\linewidth]{insertion/robot_view_screenshots/water_target_image3.png}\\
        \centering
    \end{subfigure}  \hfill
    \begin{subfigure}[b]{0.20\linewidth}
        \includegraphics[width=0.99\linewidth]{insertion/robot_view_screenshots/water_target_image4.png}\\
        \centering
    \end{subfigure}    
    \\
    \centering
    \begin{subfigure}[b]{0.40\linewidth}
        \includegraphics[width=0.99\linewidth]{figs/real_world_control_bias.pdf}\\
        \centering
        (a)
    \end{subfigure} \hfil
    \begin{subfigure}[b]{0.49\linewidth}
        \includegraphics[width=0.99\linewidth]{figs/exploration_noise.pdf}\\
        \centering
        (b)
    \end{subfigure}
    %\includegraphics[scale=1.0]{figurefile}
    \caption{Image-based training using SAC and TD3, together with residual RL and pure RL. The average returns are shown in (a). Plot (b) compares the mean final distance of the image-based training with the dense and sparse setting. }
    \label{image_figure}
    \vspace{15pt}
\end{figure}
\fi


\iffalse
\begin{table}[b!]\label{tab:use_cases}
\centering
\small
\vspace{15pt}
\caption{Evaluation of residual RL on different industrial connectors. }
\renewcommand{\arraystretch}{1.6}
\begin{tabular}{|l|c|c|c|c|l|}
\hline
\multirow{2}{*}{Success rates} & \multirow{2}{*}{P-controller} &  \multirow{2}{*}{Pure RL} &\multicolumn{3}{c|}{Residual RL} \\ \cline{4-6} 
 &  & & Full state & \multicolumn{2}{c|}{Pixels} \\ \hline
USB & 1.0  &  xxx & 0.96  & \multicolumn{2}{c|}{\cellcolor{gray!20} 1.0} \\ \hline
U-sub connector & 0.88 &  xxx &\cellcolor{gray!20} 0.92 & \multicolumn{2}{c|}{0.64} \\ \hline
Waterproof connector & 0.52 &  xxx &0.16 & \multicolumn{2}{c|}{0.48} \\ \hline
\end{tabular}
\vspace{6 pt}
\end{table}
\fi

\iffalse
\begin{figure*}[t]
    \vspace{6pt}
    \centering
    \begin{subfigure}[b]{0.32\linewidth}
        \includegraphics[width=0.99\linewidth]{figs/env_variance_sim.pdf} \\
        \centering
        (a)
    \end{subfigure}
    \begin{subfigure}[b]{0.32\linewidth}
        \includegraphics[width=0.99\linewidth]{figs/control_noise_sim.pdf} \\
        \centering
        (b)
    \end{subfigure}
    \begin{subfigure}[b]{0.32\linewidth}
        \includegraphics[width=0.99\linewidth]{figs/control_bias_sim.pdf} \\
        \centering
        (c)
    \end{subfigure}
    \caption{Simulation results for different experiments. In each plot, the final average return obtained by running the method for various settings of a parameter is shown. Plot (a) shows that residual RL can adjust to noise in the environment caused by rotation of the blocks in a range of $0$ to $0.3$\,rad. In plot (b), residual RL finds robust strategies in order to reduce the effect of control noise, as the final average return is not greatly affected by the magnitude of noise. Plot (c) shows that residual RL can compensate for biased controllers and maintains good performance as control bias increases, while the performance of the hand-designed controller dramatically deteriorates with higher control bias.} %
    \label{fig:control_bias}
\end{figure*}
\fi


\subsection{Exploration Comparison}
All experiments were also performed using TD3 instead of SAC. 
The final success rates of these experiments are included in Fig.~\ref{fig:tables}. 
When combined with residual RL, SAC and TD3 perform comparably. 
However, TD3 is often substantially less robust.
These results are likely explained by the exploration strategy of the two algorithms. 
TD3 has a deterministic policy and fixed noise during training, so once it observes some high-reward states, it quickly learns to repeat that trajectory. 
SAC adapts the noise to the correct scale, helping SAC stay robust to small perturbations, and because SAC learns the value function for a stochastic policy, it is able to handle some degree of additive noise effectively.
We found that the outputted action of TD3 approaches the extreme values at the edge of the allowed action space, while SAC executed less extreme actions, which likely further improved robustness.



\section{Conclusion}\label{sec:discussion}
In this paper, we studied deep reinforcement learning in a practical setting, and demonstrated that deep RL can solve complex industrial assembly tasks with tight tolerances.
We showed that we can learn insertion policies with raw image observations with either binary outcome-based rewards, or rewards based on on goal images.
We conducted a series of experiments for various connector type assemblies, and demonstrated the feasibility of our method under challenging conditions, such as noisy goal specification and complex connector geometries.
Reinforcement learning algorithms that can automatically learn complex assembly tasks with easy-to-specify reward functions have the potential to automate a wide range of assembly tasks, making this technology a promising direction forward for flexible and capable robotic manipulators.

There remains significant challenges for applying these techniques in more complex environments. One practical direction for future work is focusing on multi-stage assembly tasks through vision. This would pose a challenge to the goal-based policies as the background would be visually more complex. Moreover, multi-step tasks involve adapting to previous mistakes or inaccuracies, which could be difficult but should be able to be handled by RL. 
Extending the presented approach to multi-stage assembly tasks will pave the road to a higher robot autonomy in flexible manufacturing.

% \section{Acknowledgements}
% This work was supported by the Siemens Corporation,
% the Office of Naval Research under a Young Investigator
% Program Award, and Berkeley DeepDrive.


% \section{Acknowledgements}


{\small
\bibliographystyle{IEEEtran}
\bibliography{example}
}

\end{document}
