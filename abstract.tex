For robots to perform tasks in the unstructured environments of the real world, they must be able to be tasked with a desired objective in a general way, and learn to perform the desired task quickly if the robot does not already know how to accomplish it. In this thesis, we explore deep reinforcement learning as a solution to enable this vision for scalable learning-based real-world robotics through two main themes: accelerating reinforcement learning from prior data and self-supervised RL. Accelerating RL from prior data or prior knowledge is important for making reinforcement learning algorithms sufficiently sample-efficient to run directly in the real world. We discuss utilizing human demonstrations to accelerate reinforcement learning, using human-designed residual controllers in combination with reinforcement learning for industrial insertion tasks, and algorithms for offline reinforcement learning that can also benefit from a small amount of online fine-tuning.
Concurrently, while sample-efficiency of reinforcement learning algorithms is a well-appreciated problem, additional problems arise around agents that can learn from rich observations such as images: in particular, reward supervision and collecting data autonomously. We discuss self-supervised RL through goal reaching with a generative model, allowing agents to evaluate their own success at reaching goals and autonomously propose and practice skills. In the final section, we consider combining offline policy learning with self-supervised practice, allowing robots to practice and perfect skills in novel environments. These directions enable robots to supervise their own data collection, learning complex and general manipulation skills from interaction.