% The aim: general robotics
While robots are becoming ubiquitous in manufacturing and other industries, the capability of present-day robotics systems are still quite limited. They often require very specific hardware designed for a specific taskm, with manually engineered motions. In effect, they use precision and human ingenuity in order to solve the tasks they need to perform. In contrast, humans perform highly skillful dexterous manipulation very naturally. It remains a challenge to develop equivalently robust feedback controllers for robots that can adapt to a wide variety of situations to accomplish goals like cooking a meal, or even the majority of tasks that humans still do in an industrial assembly line today. How can we develop methods to do so?

% Existing work
Existing work in robotics focuses on accomplishing well-specified tasks in carefully controlled environments.

% Deep learning
Instead, the past decade of deep learning suggests that learning from large datasets is the key to such open-world generalization for robots.
Models that are pre-trained on broad datasets and can be fine-tuned to specific problems have driven recent progress in vision and NLP.
For instance, bidirection encoders from transformers (BERT)~\cite{devlin2019bert} pretrained on large language datasets (100Ms of sentences) is now the standard starting point for fine-tuning to language models on custom language tasks and on custom data of much smaller size (10,000s of sentences).
This type of pretraining capability enables machine learning to be used to solve more bespoke problems, greatly widening the applicability of ML methods to real-world applications.
Similarly, robotics applications are often narrow, so the ability to utilize broad prior data but fine-tune on a specific task would be extremely valuable.
However, in robotics, such data sharing and model sharing have proven to be relatively difficult.

% Deep reinforcement learning successes

% What challenges remain: the chicken and the egg problem between data and policy

% Challenge: goal specification

% Challenge: utilizing prior data

% Vision for robotics if these are solved

% Outline of thesis