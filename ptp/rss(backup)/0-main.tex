\documentclass[conference]{IEEEtran}
\usepackage{times}

% numbers option provides compact numerical references in the text. 
\usepackage[numbers]{natbib}
\usepackage{multicol}
\usepackage[bookmarks=true]{hyperref}

%%%%%%%%%%%%%%%%%%%%%%%%%%%%%%%%%%%%wa
% User commands.
\usepackage{mathtools}
\usepackage{amsfonts}  % for \mathbb
\usepackage{graphics}
\usepackage[pdftex]{graphicx}
\usepackage{wrapfig}
\usepackage{color}
% \usepackage[symbol]{footmisc}

% % Algorithm.
\usepackage{algorithmicx}
\usepackage[ruled]{algorithm}
\usepackage{algpseudocode}
\usepackage{amssymb}

% Caption
% \usepackage{caption}
% \usepackage{subcaption}

% Figure
\usepackage{wrapfig}

% ---------------------------------
% User defined Macros
% ---------------------------------

\algnewcommand\algorithmicforeach{\textbf{for each}}
\algdef{S}[FOR]{ForEach}[1]{\algorithmicforeach\ #1\ \algorithmicdo}

% KL Divergence 
\DeclarePairedDelimiterX{\infdivx}[2]{(}{)}{%
  #1\;\delimsize\|\;#2%
}
\newcommand{\infdiv}{\infdivx}
\newcommand{\kld}[2]{\ensuremath{D_{KL}\inf
divx{#1}{#2}}}

% Omit final dot from each def.
\usepackage{expl3}
\ExplSyntaxOn
\newcommand\latinabbrev[1]{
  \peek_meaning:NTF . {% Same as \@ifnextchar
    #1\@}%
  { \peek_catcode:NTF a {% Check whether next char has same catcode as \'a, i.e., is a letter
      #1.\@ }%
    {#1.\@}}}
\ExplSyntaxOff
\def\eg{\latinabbrev{e.g}}
\def\etal{\latinabbrev{et al}}
\def\etc{\latinabbrev{etc}}
\def\ie{\latinabbrev{i.e}}

% Vector.
\let\vec\mathbf

% Comments
\providecommand{\kuan}[1]{{\color{blue} [Kuan: #1]}}
\providecommand{\kuan}[1]{{\color{red} [Ashvin: #1]}}

%%%%%%%%%%%%%%%%%%%%%%%%%%%%%%%%%%%%

\pdfinfo{
   /Author (Homer Simpson)
   /Title  (Robots: Our new overlords)
   /CreationDate (D:20101201120000)
   /Subject (Robots)
   /Keywords (Robots;Overlords)
}

\begin{document}

% paper title
\title{
Autonomous Learning of Robot Skills via Planning 
}
%%SL.12.28: I think it's a nice starting point for the title! I guess the idea you want to get across is that the planning component is what enables the learning to be autonomous, which makes a lot of sense. There are a number of unfortunate potential misunderstandings though that we should try to avoid:
% 1. robotics people will tend to (mis)interpret "planning" as more traditional motion planning, which might make the title come across as kind of simplistic -- planning is a standard way to get autonomous robot behavior (and the autonomous *learning* part might be missed)
% 2. some people might not quite get what the "autonomous learning" bit means, or that the paper has anything to do with RL at all
% a few ideas for titles that kind of revolve around a similar theme but try to address some of these potential issues:
% - Automating Real-World Goal-Conditioned RL via Goal Planning
% - Planning with Goal-Conditioned Policies for Autonomous RL
% - Autonomous Robotic Reinforcement Learning via Goal Planning

% You will get a Paper-ID when submitting a pdf file to the conference system
\author{Author Names Omitted for Anonymous Review. Paper-ID [add your ID here]}

%\author{\authorblockN{Michael Shell}
%\authorblockA{School of Electrical and\\Computer Engineering\\
%Georgia Institute of Technology\\
%Atlanta, Georgia 30332--0250\\
%Email: mshell@ece.gatech.edu}
%\and
%\authorblockN{Homer Simpson}
%\authorblockA{Twentieth Century Fox\\
%Springfield, USA\\
%Email: homer@thesimpsons.com}
%\and
%\authorblockN{James Kirk\\ and Montgomery Scott}
%\authorblockA{Starfleet Academy\\
%San Francisco, California 96678-2391\\
%Telephone: (800) 555--1212\\
%Fax: (888) 555--1212}}


% avoiding spaces at the end of the author lines is not a problem with
% conference papers because we don't use \thanks or \IEEEmembership


% for over three affiliations, or if they all won't fit within the width
% of the page, use this alternative format:
% 
%\author{\authorblockN{Michael Shell\authorrefmark{1},
%Homer Simpson\authorrefmark{2},
%James Kirk\authorrefmark{3}, 
%Montgomery Scott\authorrefmark{3} and
%Eldon Tyrell\authorrefmark{4}}
%\authorblockA{\authorrefmark{1}School of Electrical and Computer Engineering\\
%Georgia Institute of Technology,
%Atlanta, Georgia 30332--0250\\ Email: mshell@ece.gatech.edu}
%\authorblockA{\authorrefmark{2}Twentieth Century Fox, Springfield, USA\\
%Email: homer@thesimpsons.com}
%\authorblockA{\authorrefmark{3}Starfleet Academy, San Francisco, California 96678-2391\\
%Telephone: (800) 555--1212, Fax: (888) 555--1212}
%\authorblockA{\authorrefmark{4}Tyrell Inc., 123 Replicant Street, Los Angeles, California 90210--4321}}


\maketitle

\begin{abstract}
\color{blue} [We are sorry that the manuscript is not ready for reviewing yet.]
\end{abstract}

\IEEEpeerreviewmaketitle

% \section{Introduction}

% %% What is the problem?

% Solving long horizon tasks has been an important and interesting challenge in robotics.
% Traditional methods that try to manually engineer primitive skills, then plan over them, can be brittle for two reasons.
% First, engineering the primitive skills that are robust can be difficult.
% Second, connecting the skills together can lead to compounding errors or unmet preconditions if each individual skill is not perfectly robust.
% Robust, generalizing machine learning models in domains such as computer vision and NLP has been attributed to training large models on massive datasets\cite{krizhevsky2012imagenet}.

% Learning methods have the potential to solve 
% Given a prior dataset of behaviors, how can an agent learn to compose them with a small amount of online interaction?

% %% Why is it interesting and important?

% If robots were able to successfully perform 
% Learning long-horizon goals and bootstrapping data collection

% %% Why is it hard? (E.g., why do naive approaches fail?)



% %% Why hasn't it been solved before? (Or, what's wrong with previous proposed solutions? How does mine differ?)

% The combination of conditional generative models and planning for composing skills, and offline RL to learn low-level skills

% %% What are the key components of my approach and results? Also include any specific limitations.



\section{Introduction}

% \subsection{Introduction - Reset-Free}

% %% What is the problem?
% % Large datasets for robotics -> Must be collected on-policy -> Supervision cost -> reset-free data collection
% The success of machine learning in domains such as computer vision and NLP has been attributed to training large models on massive datasets~\cite{krizhevsky2012imagenet}.
% %%SL.12.28: Reasonable way to start, but given that we are submitting this to a robotics venue, maybe the above sentence needs to be preceded by one sentence that explains why we are bothering with learning in the first place (i.e., the flow should be robots need learning to handle complexity -> learning needs large datasets -> therefore we need [whatever])
% In robotics, to be robust to a wide variety of potential situations that a robot may encounter, we analogously need large models and large datasets. Although prior work in robotics has attempted to collect and learn from large datasets~\cite{levine2017grasping} [actionable models, conservative data sharing], these datasets are usually collected manually by humans, or with heuristic policies. But for scalable robot learning, collecting data in diverse environments is exactly the same as solving tasks in diverse environments; robots will have to collect data using their prior experience to generalize and attempt coherent and useful behaviors in new environments.

% What has precluded scalable robot data collection thus far? First, human supervision needs to be provided in the form of reward supervision. This can be alleviated with self-supervised goal-conditioned methods that are general enough to represent a wide variety of tasks and do not require external reward supervision~\cite{schaul2015uva, andrychowicz2017her, nair2018rig}. But a second, and less-considered form of scaffolding that humans provide to robot learning is providing resets to the environment. Without resets, RL agents generally fail to learn, often getting trapped in one area of the state space and suffering from heavily correlated data [cite].
% %%SL.12.28: OK, these paragraphs generally have the right idea, though rhetorically it might need to be refined a bit to get this point across more clearly. Basically, right now the connection from "need lots of data" to "need autonomous RL" is not crystal clear. Perhaps we can be more forceful, eg explicitly say that as long as robotic data collection requires meticulous human oversight, robot datasets will always be limited in size, but robots are autonomous, so shouldn't they instead have an advantage in terms of collecting large datasets (autonomously)? The other problem is that the above kind of just motivates reset-free RL, it doesn't really motivate planning. I expect that comes later, but now the motivation is starting to get long-winded...

% %% Why is it interesting and important?
% % Large scale on-policy learning can be used to learn more general skills, and finetune autonomously to a test skill. %% Addressed in par 1

% %% Why is it hard? (E.g., why do naive approaches fail?)
% % Reset-free learning has not been considered much with goal-conditioned RL which is a natural fit.
% To address this issue, prior work has considered a few different strategies.
% One is to learn a backward controller to reset the environment, but this can also be difficult to learn for the same reasons that plague learning a forward controller [han 2015].
% Another is to attempt to remain in states the agent is confident in, but this can be difficult to guarantee and also slow down learning [leave no trace].
% More recently, reset-free learning has been considered in the context of multi-task or goal-conditioned reinforcement learning, where goals or tasks can be set sequentially to reset each other [abhishek and archit's recent papers].
% However, these goals are set heuristically and the policies are learned from scratch, which is not amenable for real-world robot learning.
% %%SL.12.28: This paragraph comes across as quite weak to me. It kind of just lists a bunch of prior works and then, without any justification, labels them as "heuristic" -- that's not going to be very convincing. It is however OK to defer a detailed discussion of prior work to the related work section, where we can address it in more nuance, but we need to make it very clear in the introduction why the problem is *difficult*. Perhaps this can be a good opportunity to simultaneously bring in a bit of motivation for why we want planning?

% %% Why hasn't it been solved before? (Or, what's wrong with previous proposed solutions? How does mine differ?)
% % In goal-conditioned RL, for resets planning is required.
% In this work, we show how the confluence of offline goal-conditioned RL and planning with a context-conditioned affordance model can enable reset-free learning and largely alleviate the need for human supervision in robot learning.
% %%SL.12.28: This comes out of nowhere, change the previous paragraph so that it provides a bit of motivation for goals and planning, as well as (hopefully) some offline stuff, and then this paragraph will make more sense.
% Offline RL enables policy generalization from large prior datasets to new, unseen environments, allowing for coherent behavior in these new environments.
% Context-conditioned affordance models can be utilized to sequentially set goals that are appropriate and informative from new scenarios that the agent experiences, while preventing the agent from correlated experience.
% %%SL.12.28: The goal of this paragraph should be to motivate the technical components of our approach. This would also be a good place to reference a figure that illustrates the main ideas along with a teaser.

% %% What are the key components of my approach and results? Also include any specific limitations.
% % Planning + reset free enables self-supervised learning of tasks.

% The main contribution of this work is to show that planning with a  context-conditioned affordance model enables self-supervised learning of skills with minimal human intervention.
% We show that with careful architecture choices, planning can significantly improve goal-conditioned RL beyond previously proposed approaches.
% %%SL.12.28: This makes it seem like the main contribution is to show that planning helps with goal-conditioned RL, but that's not really the contribution -- plenty of prior papers have shown that.
% We apply these findings in complex simulation environments and a real-world drawer opening and object grasping environment.
% %%SL.12.28: "apply these findings" sounds pretty weak, can we instead more explicitly state what the method accomplishes that is actually impressive?

Existing robotics methods have struggled to be robust to the vast real-world variation that robots may encounter.
%%SL.1.5: As a minor stylistic point, it can be a bit of a downer to start with a very negative statement like this off the bat. It's important to articulate why the problem is hard, but it's often better to start with a sentence about why the problem is important and then why it's hard ("X is great, but X is hard" vs "you all suck at X")
In contrast, handling real-world variation is relatively successful
%%SL.1.5: Grammar nitpick: "handling ... is successful" implies that the fact that it is handling leads to success, not that it is handling successfully (so just rephrase this to be grammatically correct). I.e., if I say "my ability to steal made me successful" it doesn't mean that I'm successful at stealing, but that the fact I can steal made me successful (at something else presumably)
in domains such as computer vision and NLP, which can be attributed to training large machine learning models on massive datasets~\cite{krizhevsky2012imagenet}.
Although prior work in robotics has attempted to collect and learn from large datasets~\cite{levine2017grasping} [actionable models, conservative data sharing]
%%SL.1.5: would be good to cite some things we are not conflicted with too?
using reinforcement learning, these datasets are usually collected manually by humans, or with heuristic policies.
For scalable robot learning, collecting data in diverse environments is exactly the same as solving tasks in diverse environments.
%%SL.1.5: I'm not sure this follows logically -- one can collect data in many ways, not all of which require solving the task. Maybe make this argument more explicit?
Instead of relying heavily on humans, robots will have to collect data using their prior experience to generalize and attempt coherent and useful behaviors in new environments.
%%SL.1.5: Generally I see the point you are trying to make, and it's an important one, but the last three sentences of this paragraph are quite clumsy, and the argument is not made in a logically coherent manner. I would suggest just rewriting the last half of this paragraph to make your argument more persuasively and more logically.

%%SL.1.5: Something that's a little awkward here: what is the problem we are solving? From the previous paragraph and the first sentence of this paragraph, one gets the impression that the problem is "scalable autonomous robot data collection." But if that's the problem we are solving, then what is the evidence we intend to present illustrating that we've done this? Are we going to experimentally validate that our method is more scalable than some other methods for data collection? That it collects more data or better data? Something here feels a bit disconnected...
What has precluded scalable autonomous robot data collection thus far?
First, much of the prior work in RL requires online data collection.
Using offline RL algorithms that can utilize static datasets, as in other deep learning fields, mitigates this issue.
%%SL.1.5: Above sentence is quite awkwardly phrased -- other deep learning fields don't use offline RL algorithms
Second, human supervision needs to be provided in the form of rewards.
This can be alleviated with self-supervised goal-conditioned methods that are general enough to represent a wide variety of tasks and do not require external reward supervision~\cite{schaul2015uva, andrychowicz2017her, nair2018rig}.
%%SL.1.5: So it's not a problem? Then why has it precluded scalable autonomous robot data collection?
Third, an under-considered form of scaffolding that humans provide to robot learning is providing resets to the environment.
%%SL.1.5: I think there a simpler and more concise way to phrase the above sentence
Without resets, RL agents generally fail to learn, often getting trapped in one area of the state space and suffering from heavily correlated data [cite].
To scale up robot learning in the real world, we need to enable the robot to effectively explore and reset the environment with minimum human intervention.
%%SL.12.28: OK, these paragraphs generally have the right idea, though rhetorically it might need to be refined a bit to get this point across more clearly. Basically, right now the connection from "need lots of data" to "need autonomous RL" is not crystal clear. Perhaps we can be more forceful, eg explicitly say that as long as robotic data collection requires meticulous human oversight, robot datasets will always be limited in size, but robots are autonomous, so shouldn't they instead have an advantage in terms of collecting large datasets (autonomously)? The other problem is that the above kind of just motivates reset-free RL, it doesn't really motivate planning. I expect that comes later, but now the motivation is starting to get long-winded...
%%SL.1.5: The planning part is still not getting motivated well, but I do think the above paragraphs improved a bit. I think you have generally the right idea in terms of the argument to make, but the argument needs to be stronger and more logical, and it needs to more directly motivate the particular thing that we are doing in this paper.

% To address this issue, prior work has considered self-supervised goal-conditioned RL which requires relatively little human supervision. Although this framework does not immediately solve the reset problem,
Reset-free learning has been considered more recently in this context of multi-task or goal-conditioned reinforcement learning, where goals or tasks can be set sequentially to reset each other \cite{Gupta2021ResetFreeRL, Lu2021ResetFreeLL} [abhishek and archit's recent papers].
% One is to learn a backward controller to reset the environment, but this can also be difficult to learn for the same reasons that plague learning a forward controller [han 2015].
% Another is to attempt to remain in states the agent is confident in, but this can be difficult to guarantee and also slow down learning [leave no trace].
% More recently, reset-free learning has been considered in the context of multi-task or goal-conditioned reinforcement learning, where goals or tasks can be set sequentially to reset each other [abhishek and archit's recent papers].
However, these goals are set heuristically and the policies are learned from scratch, which is not amenable for real-world robot learning.
To make autonomous practicing skills minimally expensive, the robot would ideally require only a small amount of initial supervision on what skills to practice, and little supervision as the agent trains.
One way to accomplish this is for the human to provide a small amount of initial and goal states, and the robot to attempt to traverse back and forth between these states [han 2015].
% can practice starting with an offline dataset to bootstrap the skills from and \textit{planning to practice} various skills by traversing between these states.
However, as we show in our experiments, the robot may reach arbitrary and unexpected states that existing goal-conditioned offline RL cannot recover from.
In this work, we investigate how planning over subgoals can alleviate the reset problem to enable robots to automatically learn to solve novel tasks
%%SL.12.28: I think the "solve novel tasks" bit can get us in trouble -- is it actually solving *novel* tasks?
with minimum human intervention.
That is, although the robot may not have perfect skills to begin with and may reach arbitrary and unexpected states, it can \textit{plan to practice}, optimizing a sequence of subgoals in order to practice skills while resetting itself.
%%SL.1.5: I think the above paragraph is too much of a "related work" paragraph that tackles a few individual prior works and talks about why they don't solve the problem. But it's not entirely clear what the problem is or what we are proposing to do at this stage. It would be more effective to rewrite this paragraph and instead use it to motivate what we will be doing -- basically, the above paragraph needs to convince the reader that an algorithm based on goal-conditioned RL and planning will address the challenges laid out in the first two paragraphs.

% % Challenges of RL in robotics: Reward shaping and episodic reset.  
% Reinforcement learning holds the promise of enabling robots to automatically acquire useful skills through interactions in a given environment. However, the effectiveness of applying such trial-and-error paradigms to robotics relies on requirements that involve intensive human labor and expertise. First, the learning process requires informative reward signals that are usually carefully devised or annotated by experts. Second, the agent are usually trained in an episodic manner and needs to be reset to an initial state after each episode, which can require nontrivial manual labor in the real world and expensive computation in the simulation. Such reward shaping and episodic resets often need to be instrumented differently in each new task, which becomes a major bottleneck for applying reinforcement learning in challenging robotics domain. To scale up robot learning in the real world, we need to enable the robot to effectively explore and reset the environment with minimum human intervention. 
% %%SL.12.28: I think the first paragraph for this version of the intro can mostly mirror the design of the previous version of the intro -- it's still important to motivate that we need lots of data and autonomy is a bottleneck in that, though I do think it's nice to start bending the narrative toward planning (by motivating the challenges it eventually will address) right here in the first para the way you did.

% Previous works.
% To combat this challenge, a plurality of methods has been developed in the robot learning domain.
% %%SL.12.28: above sentence kind of doesn't say much, delete?
% Various exploration strategies have been designed to encourage the state coverage by an agent~\citep{NIPS2016_6591, pathakICMl17curiosity, burda2018exploration, conti2018improving} in hard-exploration problems~\citep{maillard2014hard}. Naively covering intermediate states can be insufficient for the agent to connect the dots and discover the final solution. In complicated tasks, it could also be difficult to visit diverse states by directly exploring in the given environment. Recent works have also aimed to avoid frequent resets of the environment using predefined task graphs~\citep{Gupta2021ResetFreeRL}, curriculum learning~\citep{Lu2021ResetFreeLL}.
% % , and conservative exploration~\citep{}. 
% These methods either require either expert knowledge or learning reset-free exploration from scratch. 
% % \kuan{Need a better summary and comparison to the prior work.}
% %%SL.12.28: I'm not sure what the citations to the exploration literature have to do with the reset-free papers. In general, the above paragraph is rather hard to follow -- the point of a paragraph like this is to explain why the problem is hard and why prior work has not solved it, but this is not apparent to me from reading the above.

% Problem Statement.

% In this work, we investigate how planning over subgoals can alleviate the reset problem to enable robots to automatically learn to solve novel tasks
% %%SL.12.28: I think the "solve novel tasks" bit can get us in trouble -- is it actually solving *novel* tasks?
% with minimum human intervention.
% That is, although the robot may not have perfect skills to begin with and may reach arbitrary and unexpected states, it can optimize a sequence of subgoals in order to practice skills while resetting itself.
% Instead of training the policy from scratch, we expect the robot to make use of prior data that contains expert demonstrations of related skills in the same environment.
% When training the robot to solve a novel task, we would like the robot to extract transferable knowledge from such prior data to reduce the required human interventions for reward shaping and episodic resets
% In particular, we consider a goal-conditioned reinforcement learning problem setup where each task is specified by a set of initial states and a set of goal states in an environment. 
% %%SL.12.28: Generally, this is a good paragraph, and it's important to basically have a paragraph like this, but the problem is that it doesn't connect very well to the method -- the goal-conditioned bit kind of comes out of nowhere at the end, and there is no motivation for planning at all. Perhaps somewhere before this paragraph (or at the beginning of this paragraph?) we can add some more discussion that motivates *why* we goal conditioned RL and planning can help with autonomy, and then we can summarize what we do (perhaps with the help of a Figure 1 diagram).
%%AVN I split this paragraph into the top and bottom paragraphs, but if we liked having this ordering of the information we can restore it


% Technical Solution.
To this end, we propose MODEL\_NAME, an approach that autonomously learns to solve novel tasks via learning a goal-conditioned policy from offline data and then optimizing over a sequence of subgoals to collect exploration data autonomously.
%%SL.12.28: The term "planning" won't mean the same thing to your readers taht it means to you, perhaps better explain it as something like learning a goal-conditioned policy and then optimizing over a sequence of goals to issue to this policy
We consider a goal-conditioned reinforcement learning problem setup where each task is specified by a set of initial states and a set of goal states in an environment. 
When training the robot to solve a novel task, we would like the robot to extract transferable knowledge from such prior data to reduce the required human interventions for reward shaping and episodic resets.
In our approach we extract two types of models from the prior data.
First, a goal-conditioned policy is learned offline 
% pretrained by a cutting-edge
%%SL.12.28: doesn't seem relevant that it's cutting edge, since that's not our cnotribution?
% offline reinforcement learning algorithm~\cite{kostrikov2021offline} 
and then finetuned to solve a novel task through online exploration.
Second, we train a generative model that learns to capture the distribution of meaningful subgoals
%%SL.12.28: if it's key to our approach, we should have motivated it in one of the preceding paragraphs
conditioned on the current state from the prior data.
When the robot explores in the novel task, a planner uses the generative model to propose subgoals and sequence them into an informative plan to guide the robot to traverse between the initial states and the goal states.
To reduce the need of episodic resets, our approach iterates between planning towards the initial states and the goal states periodically.
%%SL.1.5: It makes sense to have a paragraph like this that essentially explains our solution, but this should follow a paragraph that motivates the parts (basically, when the reader reads this paragraph, it should already be clear to them why this is a good idea). Additionally, it would be really nice if these design decisions clearly connect to the challenges we articulated in the first two paragraphs -- right now, the bit about resets is kind of an afterthought, and it's not clear how any of this will actually address the problem that the first two paragraphs lay out. This paragraph would also be a good place to reference a figure.
% A goal relabeling mechanism~\citep{andrychowicz2017hindsight} is designed to scaffold the training of the goal-conditioned policy using the collected trajectories guided by the plan. 

%%SL.12.28: State key contributions and then key experimental takeaways.
% Experiments.
The main contribution of this work is to show that planning with a  context-conditioned affordance model
%%SL.1.5: I don't think the notion of an affordance model came up in the previous paragraphs
enables self-supervised learning of skills with minimal human intervention.
%%SL.1.5: I think this is a *very* strong claim that our experiments will fail to back up. Generally, when people see "self-supervised" they mean the robot decides on the tasks -- but in our method we still specify which goals to accomplish and provide prior data that is essentially demonstration data. I think it's a long shot to claim that this is really "self-supervised".
We demonstrate that our approach can efficiently learn to solve unseen
%%SL.1.5: what does it mean to "learn to solve unseen ... tasks"? it kind of sounds like the robot has to perform the task without seeing what it is or something, a bit of a non sequitur
robotic manipulation tasks in simulation and the real world without meticulous reward shaping and with a limited number of episodic resets.
Compared to existing baselines, our approach achieves higher task success rates after the same amount of exploration steps and resets.
%%SL.1.5: The problem is that the above discussion of results is very unclear about the problem setting and the assumptions. This is a problem, because then readers will expect things that you can't deliver. Need to be clearer about precisely what is being solved and how. For example, it's not even clear that prior data is being used, etc.

%%SL.12.28: Generally, I do think the more planning-centric motivation is a bit better, but this version of the intro needs a lot of work to more thoroughly motivate the various design choices.

\section{Related Work}
% \kuan{This section is outdated. We will finish it soon.}

%%SL.2.24: I take a closer look through this section once it's revised, but a very important thing to keep in mind: If you are submitting the paper to a robotics conference, the related work section needs to touch on the most relevant *robotics* work, not just a bunch of deep RL papers. This includes a brief summary of robotic RL papers (could just be a sentence), and possibly also a short discussion of multi-task learning in RL. Try not to just cite things from the past three years, the robotics literature on multi-task learning goes quite far back. It might also help to have a sentence or two about planning in robotics (not planning in the deep RL literature, but actual robotics literature, like Likhachevsky, Kaelbling, Lozano-Perez, etc.) and explain how that's different from what is going on in our work.
%%SL.2.27: For a 6-page paper, we really can't have related work go over the second page. Ideally, Sec I+II would be 1.75 pages max, else we just don't have enough space to explain the method. Try to reduce text and keep citations. More generally, once the particulars are cleaned up, we really need to think about related work strategy. There are a few papers that are extremely relevant, and we need to make it 110% clear why our work is novel relative to these papers (that is really the main thing that matters, the rest is just making sure to cite enough prior work that your reviewers don't get offended about the paper missing some big chunk of prior work)
%%AVN.2.28 looks like we have 8 pages so we don't have to worry so much about space

We propose to use a combination of optimization-based planning and fine-tuning with goal-conditioned reinforcement learning from prior data in order to allow robots to learn temporally extended skills.
In this section, we cover prior methods in offline RL, planning, goal-conditioned RL,
% , offline RL, planning, 
and how they relate to our method.
%%SL.2.27: Maybe delete the above (it just follows from intro), remove paragraph headings, and contract the paragraphs below.

\textbf{Learning from prior data.}
Offline reinforcement learning methods learn from prior data~\cite{lange2012batch, fujimoto2019off, kumar2019stabilizing, zhang2021brac, kumar2020conservative, fujimoto2021minimalist,singh2020cog}, and can also finetune through online interaction~\cite{nair2020awac, villaflor2020finetuning, lu2021awopt, Khazatsky2021WhatCI, lee2021finetuning, meng2021starcraft}. Such methods have been used in a variety of robotic settings~\cite{kalashnikov2018scalable,cabi2019scaling,kalashnikov2021mtopt,lu2021awopt}. Our focus is not on introducing new offline RL methods. Rather, our work shows that planning over subgoals for a goal-conditioned policy that is pretrained offline can enable finetuning for temporally extended skills that would otherwise be very difficult to learn.
%Recent advances in offline reinforcement learning have made robotic learning from prior data using RL practical.
%%SL.3.1: It would be good to add citations to actual robotics papers -- right now it seems like very few of the papers cited in this paragraph actually pertain to robotics.
%Offline RL methods usually add conservatism to the RL objective to make training without collecting additional data more stable~\cite{lange2012batch, fujimoto2019off, kumar2019stabilizing, zhang2021brac, kumar2020conservative, fujimoto2021minimalist}. These methods have been applied in robotic domains to learn skills, including goal-conditioned skills from off-policy data~\cite{kalashnikov2021mtopt, yu2021conservative, chebotar2021actionable}. However, less work has addressed the problem of effectively collecting additional data online for improvement after pre-training on the prior data~\cite{nair2020awac, villaflor2020finetuning, lu2021awopt, Khazatsky2021WhatCI, lee2021finetuning, meng2021starcraft}.
%%SL.3.1: I don't think it's relevant that there is "less" work. It's more important to explain how what we are doing is distinct. In general, I would recommend mostly rewriting this paragraph to focus more on how this all compares to what we are doing rather than having a bunch of red herrings about conservatism. For example, could structure it something like this: Offline reinforcement learning methods learn from prior data~\citep{}, and can also finetune through online interaction~\citep{}. Such methods have been used in a variety of robotic settings, including manipulation~\citep{} and navigation~\citep{}. Our focus is not on introducing new offline RL methods. Rather, our work shows that planning over subgoals for a goal-conditioned policy that is pretrained offline can enable finetuning for temporally extended skills that would otherwise be very difficult to learn.
%In this work we focus on how to fine-tune goal-conditioned policies for long-horizon tasks with minimal human supervision. 
% In the long-horizon case, this requires the use of ideas from sampling-based planning, which we will discuss next.
%%SL.2.27: I don't think we need so much text about offline RL, because it's not really the focus of this work. I also don't think this should come first. I think a better way to do it would be to have a short paragraph at the end of the related work section that basically says "we use offline RL [lots of citations] and we fine-tune [lots of citations] but we don't do anything novel here and instead just use it as a tool" (or something like that, though of course less colloquially)
%%AVN.2.28 The way the intro is written now, I think it actually makes more sense for this to come first - let me know what you think

\textbf{Goal-conditioned reinforcement learning.} The aim of goal-conditioned reinforcement learning (GCRL) is to control the agent to efficiently reach specified goal states~\cite{Kaelbling1993LearningTA, Schaul2015UniversalVF, Eysenbach2021CLearningLT}. Compared to policies that are trained to solve a fixed task, the same goal-conditioned policy can perform a variety of tasks when it is commanded with different goals. Such flexibility allows GCRL to better share knowledge across different tasks and make use of goal relabeling techniques to improve the sample efficiency without meticulous reward engineering~\cite{Andrychowicz2017HindsightER, Pong2020SkewFitSS, Fang2019CurriculumguidedHE, Ding2019GoalconditionedIL, Gupta2019RelayPL, Sun2019PolicyCW, Eysenbach2020RewritingHW, Ghosh2021LearningTR}. Prior has explored various strategies for proposing goals for exploration~\cite{nair2018rig, Nair2019ContextualIG, Khazatsky2021WhatCI, ChaneSane2021GoalConditionedRL}, and studied goal-conditioned RL from offline data~\cite{chebotar2021actionable}. However, such works generally aim to learn short-horizon behaviors, and learning to reach goals that require multiple stages (e.g., several manipulation primitives) is very difficult, as shown in our experiments. Our work aims to extend model-free goal-conditioned RL methods by incorporating elements of planning to enable effective finetuning for multi-stage tasks.

\textbf{Planning.} A wide range of methods have been developed for planning in robotics. At the most abstract level, symbolic task planning searches over discrete logical formulas to accomplish abstract goals~\cite{fikes1971strips}. Motion planning methods solve the geometric problem of reaching a goal configuration with dynamics and collision constraints~\cite{Kavraki1996, koenig2002dstarlite,  karaman2011rrtstar, zucker2013chomp, kalakrishnan2011stomp}. Prior methods have also considered task and motion planning as a combined problem~\cite{srivastava14tamp}. These methods generally assume high-level structured representations of environments and tasks, which can be difficult to actualize in real-world environments. Since in our setting we only have image inputs and not structured scene representations, we focus on methods that can handle raw images for observations and task specification.
%%SL.3.1: The above paragraph is great! I really like how it reads, very nice coverage of prior work and really crisp articulation of how it differs from ours. But still need to address the comment below.
%%SL.2.27: feels like there must be other planning-style methods that we're missing, eg the neurosymbolic programming stuff, etc. -- don't need lots of discussion about it, just a sentence with citations, but it's good to note all the stuff that does something long-horizon

%%SL.3.1: Rewrote the discussion of prior subgoal optimization methods, please reread and check this
\textbf{Combining goal-conditioned RL and planning.}
A number of recent works have sought to integrate concepts from planning with goal-conditioned policies in order to plan sequences of subgoals for longer-horizon tasks~\cite{Nasiriany2019PlanningWG, Eysenbach2019SearchOT, fang2019cavin, Charlesworth2020PlanGANMP, Pertsch2020LongHorizonVP, Sharma2021AutonomousRL, Zhang2021CPlanningAA}. These prior methods either propose subgoals from the set of previously seen states, or directly optimize over subgoals, often by utilizing a latent variable model to obtain a concise representation of image-based states~\cite{nair2018rig,ichter2018learning,nair2019hierarchical,Nasiriany2019PlanningWG,Pertsch2020LongHorizonVP, Khazatsky2021WhatCI, ChaneSane2021GoalConditionedRL}. 
The method we employ is most closely related conceptually to the method proposed by Pertsch et al.~\cite{Pertsch2020LongHorizonVP}, which also employs a hierarchical subgoal optimization, and the method proposed by Nasiriany et al.~\cite{Nasiriany2019PlanningWG}, which also optimizes over sequences of latent vectors from a generative model. Our approach makes a number of low-level improvements, including the use of a conditional generative model~\cite{Nair2019ContextualIG}, which we show leads to significantly better performance. More importantly, our method differs conceptually from these prior works in that our focus is specifically on utilizing subgoal optimization as a way to enable finetuning goal-conditioned policies for longer-horizon tasks. We show that it is in fact this capacity to enable effective finetuning that enables our method to solve more complex multi-stage tasks in our experiments.

%%SL.3.1: All the stuff below is old material for addressing planning, which I rewrote in the paragraph above.
%The main challenge for such planning methods is to effectively propose and rank achievable subgoals that lead to the final goal. Prior works either sample from the set existing states from the agent's past experience or generate unseen states from a latent space,
%using hand-designed metrics or learning value functions from scratch to rank the feasibility of the plans. In contrast to prior work, our approach learns a generative model~\cite{Nair2018VisualRL, nair2019hierarchical, Nair2019ContextualIG, Khazatsky2021WhatCI, ChaneSane2021GoalConditionedRL}
%to propose imagined goals conditioned on the observed state, by capturing the distribution of transitions in the prior data. Using the generative model to recursively propose subgoals, our approach is able to effectively propose feasible plans in novel scenarios.

%These methods for planning over images broadly involve learning latent representations to plan in a more manageable space~\cite{finn2017deepvf, coreyes2018sectar}.
%Ichter et al. use a conditional variational autoencoder (CVAE)~\cite{kingma2014vae, sohn2015cvae} to learn a generative model to draw collision-free samples from the action space~\cite{ichter2018learning}. This idea was further extended for collision-free motion planning \cite{ichter2019robot}. The main notable difference is that most of these methods represent the probability distribution of a single action mode, where the data is often deliberate for the task. 

%Closest to our method, methods have been developed that learn goal-conditioned subgoals to accomplish long-horizon tasks from image input.
%We empirically evaluate these methods in our experiments.
%Pertsch et al. learn a goal-conditioned predictor (GCP) which recursively predicts intermediate subgoals between the initial state and goal state~\cite{Pertsch2020LongHorizonVP}.
%Our method, which uses a generative model to generate plans, can \textit{compose} skills, which GCP cannot due to learning a model that is conditioned on the final state.
%Additionally, GCP uses an inverse model instead of a goal-conditioned policy as the lower-level controller, and empirically performs worse than PTP.
%Nasiriany et al. introduce latent embeddings for abstracted planning (LEAP) which uses optimization-based planning with a variational auto-encoder to handle planning over raw images~\cite{Nasiriany2019PlanningWG}.
%Our method uses the same idea, but using a conditional goal-generation model for planning allows us to plan in visually complex real-world environments, and we demonstrate empirically that our method can handle more complex scenes than LEAP.

% While, the hierarchical dynamics model in the CAVIN Planner decouples the model learning into latent code for effects and motion codes, each of which can guide the action sampling. And finally the consistency action sampling is ensured through dynamics prediction over a self-supervised dataset. 

% \kuan{Remove the non-episodic RL part and add using planning/compositionality for RL.}

% \kuan{Maybe we should merge the three sub-sections and make it more concise, since we are gonna talk about some of these topics in more details in the Background section.}
% \kuan{Maybe we should also discuss prior work on model-free RL augmented with model-based approaches.}

% without human-provided resets. We show that such a reset-free finetuning process with offline initialization can still be effective with planning and the appropriate design decisions.
%%SL.1.5: OK, this paragraph is also very important, but it also suffers from the same problem as the previous one -- there are great citations, but the reader is left unsure how what we are doing differs from these other papers that do offline pretraining + online finetuning. An additional issue is that the intro doesn't really make it crystal-clear that we are using prior data (nor does it really motivate it) -- that's not a problem with this paragraph, but rather a problem with the intro, but it really needs to be fixed.

% \textbf{Non-Episodic Reinforcement Learning.} In the canonical reinforcement learning formulation, the episodic reset enables the agent to breaks the collected experiences into episodes and periodically start over from a state sampled from the initial state probability. Without the episodic reset, the agent would need to autonomously return to a legitimate initial state by itself and avoid getting stuck in the sink states that are hard to recover from \cite{Lu2019AdaptiveOP, CoReyes2020EcologicalRL, Lu2021ResetFreeLL, Sharma2021AutonomousRL}.
% Gupta et al. propose to use a predefined task graph to manual rests performed by humans~\cite{Gupta2021ResetFreeRL}. In the task graph, the terminal state of one task can serve as the initial state of another task. Instead of explicitly utilizing any prior knowledge of the tasks, our approach avoids episodic resets by learning to plan trajectories between the initial states and the goal states.
% %%SL.1.5: it's not clear why planning trajectories avoids the need for resets (also, our model doesn't really "learn to plan" -- it learns, and then plans)
% Similarly, \cite{Nasiriany2019PlanningWG} and \cite{Sharma2021AutonomousRL}
% %%SL.1.5: here and elsewhere, use \citet instead of \cite if you want the parenthetical citation to serve as a noun
% also conduct non-episodic reinforcement learning through planning. Given the learned skill policy and dynamics model, \cite{Nasiriany2019PlanningWG} searches for trajectories that will lead to high cumulative rewards. \cite{Nasiriany2019PlanningWG} focuses on tasks that do not have an explicit goal state and the agent can keep exploring the environment as long as it does not get stuck in the sink states.
% %%SL.1.5: kind of unclear how that's different from what we're doing
% In the goal-reaching tasks, however, the agent would need to turn around after reaching the goal. \cite{Sharma2021AutonomousRL} assumes that the agent starts from the goal state and asks the agent to traverses between the final goal and a subgoal state sampled from its past experiences. It reduces the necessity of episodic resets by progressively choosing the subgoals that are closer the initial state in a curriculum learning manner. In contrast, our approach does not make such assumptions and can generate feasible subgoals that are unseen by the agent. 
% %%SL.1.5: I think we can do better in explaining how our method is different (but let's discuss more on Slack)


% \input{3-problem-statement}
\section{Planning to Practice}
%%SL.2.24: It would be good to use a more inspiring section name (e.g., name it after the method or something like that). You might also consider separating the technical section into a general section about the conceptual method, and then a separate section with implementation details to explain how the method is instantiated
%%KF.3.1: Fixed

% Our aim is to study how we could enable a robot to efficiently learn to solve novel tasks by utilizing prior data of related tasks. 
% through compositionality. In this section, we outline an approach that facilitate online fine-tuning by planning with generated subgoals. 

% We propose Planning to Practice (PTP), an approach that efficiently fine-tunes a goal-conditioned policy to solve novel tasks.  
% by exploiting the compositional structure of the demonstration data.
%%SL.2.24: I think we need to be very clear about whether it's a method for learning from demonstrations, vs an RL method. Right now, the motivation mostly presents the method as an RL method, but we start throwing "demonstration" around, then people are likely to think it's an imitation learning method. We probably want to be clear on this point and avoid muddying the waters.
% In this section, we describe how online fine-tuning can be facilitated by using subgoals and provide a recipe for effectively searching for feasible plans by composing generated subgoals in the latent space.

% ----------------------------------------------------------------------------------------
% \subsection{Online Fine-Tuning by Composing Goals}

% Utilizing prior data: offline pre-training + online fine-tuning.
    % - Our goal: Fast adaptation to novel tasks
    % - Prior data
    % - How prior data can help
    % - Why prior data is not enough: Distribution shift + unseen combination of skills
    % - Offline pre-training using IQL
    % - Online fine-tuning 
% Our approach learns a goal-conditioned policy $\pi_{\theta}(a | s, s_g)$ for solving the target task specified by a desired final goal $s_g$. The goal-conditioned policy is pre-trained on a pre-collected offline dataset $\mathcal{D}_\text{offline}$ and then fine-tuned to reach $g$ by continuously collecting online data in the replay buffer $\mathcal{D}_\text{online}$. We expect $\mathcal{D}_\text{offline}$ to contain expert demonstrations of controlling the robot to perform diverse short-horizon interactions with the objects in the environment. \kuan{Add.}
% Through pre-training on such prior data, the policy $\pi$ could learn primitive behaviors that are potentially useful for solving the target tasks. Directly fine-tuning the policy to solve the target tasks can often lead to poor performance in the target task due to three major challenges. First, we do not assume to have access to unlimited amount of prior data and the under-trained policy often struggles to make any progress. Second, the policy needs to adapt to the distributional shift~\cite{} when the environment is initialized with object arrangements or lighting conditions unseen in the prior data. Third, we would like to robot to adapt to target tasks involving more complicated multi-stage interactions with the environment, thus the robot needs to learn novel strategies to compose the primitive behaviors across a longer time horizon.

%%SL.2.24: This is a bit abrupt right now, with lots of technical concepts provided without initial context for how they constitute a complete system. What I would suggest is to expand Section III to provide a problem statement, and then at the top of Section IV provide a high-level overview (perhaps with a diagram) that explains what are the constituent parts of the full system. Ideally these two things would introduce many of the symbols, such that when these symbols are used in this subsection, it won't be for the first time.
% Propose subgoals to facilitate online fine-tuning.
    % - Using subgoals to guide the fine-tuning.
    % - Switch subgoals.
    % - Summarize why this can help
We propose Planning to Practice (PTP), an approach that efficiently fine-tunes a goal-conditioned policy to solve novel tasks.  
To enable the robot to efficiently learn to solve the target task, we propose to use subgoals to facilitate the online fine-tuning of the goal-conditioned policy. Given the initial state $s_0$ and the goal state $s_g$, we search for a sequence of $K$ subgoals $\hat{s}_1:K = \hat{s}_1, ..., \hat{s}_K$ to guide the robot to reach $s_g$. Such subgoals will inform the goal-conditioned policy $\pi(a | s, s_g)$ what is the immediate next step on the path to $s_g$ and provide the policy more dense reward signals compared to directly using the final goal. We choose the sequence of subgoals at the beginning of each episode and feed the first subgoal in the sequence to the goal-conditioned policy. The policy will switch to the next subgoal in the sequence when the current subgoal is reached or the time budget assigned for the current subgoal runs out.

The main challenge is to search for a sequence of subgoals that can lead to the desired final goals while ensuring each subgoal is a valid state that can be reached from the previous subgoal. Particularly when the states correspond to full images, most vectors will not actually represent valid states, and indeed na\"{i}vely optimizing over image pixels may simply result in out-of-distribution inputs that lead to erroneous results when input into the goal-conditioned policy.
%The space of possible subgoal sequences is $|\mathcal{S}|^ K$ dimensional, which will make it computationally intractable to find the suitable subgoals for high-dimensional state spaces and long time horizons.

% Sampling-based planning
    % Overview
    % Sample trajectories using the affordance model
    % Choose the plan that corresponds to the lowest cost.
    % Considerations and challenges.
% Given the initial state $s_0$ and the goal state $s_g$, we use a planner to search for the sequence of subgoals through sampling-based planning. The planner aims to find the optimal sequence of subgoals $\hat{s}_{1:K}^*$ using the cost function $c(s_0, \hat{s}_{1:K}, s_g)$. To find the plan, we first sample $N$ sequences of subgoals $\hat{s}_{1:K}^1, ..., \hat{s}_{1:K}^N$ from the state space and evaluate the cost for each sequence. The sequence that corresponds to the lowest cost will be chosen as $\hat{s}_{1:K}^*$ for the goal-conditioned policy. 

%%SL.2.24: I would not refer to this as a planner habitually, because it doesn't really have the structure most would associate with a planning algorithm. I think it's OK (if you *really* want to) to call it a planner *after* explaining what it is, and then saying that it's a kind of planner. But definitely don't start calling it a planner out of the gate before explaining how it works, because it is very atypical for a planning method to work like this.
%%KF.3.1: Re-organized the paragraph as below.

As outlined in Fig.~\ref{fig:intro}, we devise a method to effectively propose and select valid subgoal sequences to guide online fine-tuning by means of a generative model. At the heart of our approach is a conditional subgoal generator $g(\cdot | s_0)$ that recursively produces candidate subgoals in a hierarchical manner conditioned on the initial state $s_0$. To find the optimal sequence of subgoals $\hat{s}_{1:K}^*$, we first sample $N$ candidate sequences $\hat{s}_{1:K}^1, ..., \hat{s}_{1:K}^N$ from the state space using the conditional subgoal generator. Then we rank the candidate sequences using a cost function $c(s_0, \hat{s}_{1:K}, s_g)$. The sequence that corresponds to the lowest cost will be selected as $\hat{s}_{1:K}^*$ for the goal-conditioned policy. Through this sampling-based planning procedure, we choose the subgoal for guiding the goal-conditioned policy $\pi$ during online fine-tuning. The overall algorithm is summarized in Algorithm~\ref{algo:ptp}. Next we describe the design of each module in details.

% Both the goal-conditioned policy and the conditional subgoal generator are pre-trained on the offline data. During the online fine-tuning, we use the planner to produce subgoals for the goal-conditioned policy.

% \kuan{Fix this.}.
% Both the goal-conditioned policy and the planner
% %%SL.2.24: goal-conditioned planner and planner? is there a typo here?
% %%AVN.2.26 fixed
% are pre-trained on the prior data. During the online fine-tuning, we use the planner to produce subgoals for the goal-conditioned policy. Through hindsight experience replay~\cite{}, the goal-conditioned policy is trained to reach not only the provided subgoals but also goals that it could achieve in the later stage of the episode. Eventually, the policy is supposed to learn to efficiently reach to distant goals even when the provided plans are sub-optimal. We evaluate the trained policy by directly feeding in the final goal in each target task.
%%SL.2.24: Organizationally, I wonder if it might be a good idea to have separate subsections to discuss the planner vs online finetuning. Perhaps logically, the better way to do it is to have an overview at the top, then explain latent space goal reaching, then talk about how its hard and explain affordances and planning, and then finetuning? Otherwise the explanation of planning prior to explaining affordances, latent spaces, or anything else is a bit hard to understand, and also doesn't actually fully explain our method, since many important details (e.g., the latent space and affordances) are omitted here and then "retrofitted" into the approach in later subsections.

% ----------------------------------------------------------------------------------------
\begin{algorithm}[t]
\caption{Planning To Practice (PTP)}
\begin{algorithmic}[1]
\Require set of final goals $\mathcal{G}$, time horizon $T$, offline data $\mathcal{D}_\text{offline}$, number of subgoals $K$.

\State Train $\pi(a | s, s_g)$ and $g(s, z)$ on $\mathcal{D}_\text{offline}$.
\State Initialize the online replay buffer $\mathcal{D}_\text{online} \leftarrow \varnothing$.

\While{not converged}
    \State Reset the environment and observe $s_0$.
    \State Sample $s_g$ from $\mathcal{G}$.
    \State Plan for the subgoals $\hat{s}_{1:K}$.
    
    \State $k \leftarrow 1$
    \For{$t = 1, ..., T$}
        \State Compute the action $a_t \leftarrow \pi(a_t | s_t, \hat{s}_K)$
        \State Observe the state $s_{t+1}$ and the reward $r_t$
        \State $\mathcal{D}_\text{online} \leftarrow \mathcal{D}_\text{online} \cup (s_t, a_t, r_t, s_{t+1})$.
        
        \If{$t \pmod {\Delta t} == 0$ \textbf{or} $|| s_{t+1} - \hat{s}_K || < \epsilon$}
            \State $k \leftarrow \min(k + 1, K)$ 
        \EndIf
    \EndFor
    
    \State Train $\pi$ on batches sampled from $\mathcal{D}_\text{offline}$ and $\mathcal{D}_\text{online}$.
    
\EndWhile

\end{algorithmic}
\label{algo:ptp}
% \vspace{-5mm}
\end{algorithm}


% ----------------------------------------------------------------------------------------
\subsection{Conditional Subgoal Generation}

% Use generative model to propose reachable goals.
    % Motivation: A way to efficiently generate diverse, high-fidelity, feasible sequences of subgoals for sampling-based planning. 
    % - What the affordance model is used for
    % - Recursive generation of subgoals
    % - A sequence of latent codes can be converted to a sequence of subgoals
    % - Propose goals at multiple timescales
    
%%SL.2.24: As written, it's not entirely clear what problem is being addressed -- it would be better if we can start the section by posing a question or stating the problem that actually requires this machinery.

The effectiveness of our planner relies on the generation of diverse and feasible sequences of subgoals as candidates. Specifically, we would like to generate the candidates by sampling from the distribution of suitable subgoal sequences $p(\hat{s}_1, ..., \hat{s}_K | s_0)$ conditioned on the initial state $s_0$. Most existing methods
%%SL.2.24: Not really clear which methods this is referring to.
%%KF.3.1: Fixed.
independently sample the subgoal at each step from a learned prior distribution~\cite{Pertsch2020LongHorizonVP} or a replay buffer~\cite{Eysenbach2019SearchOT}, which is unlikely to propose useful plans for tasks with large, combinatorial state spaces (i.e., with multiple objects).

We propose to break down $p(\hat{s}_1, ..., \hat{s}_K | s_0)$ into $p(\hat{s}_1 | s_0) \Pi_{i=1}^k p(\hat{s}_i | \hat{s}_{i - 1})$ through modeling the conditional distribution $p(s' | s)$ of the reachable next subgoal $s'$. By utilizing temporal compositionality, the conditional subgoal generation paradigm improves generalization and enables generation of sequences of arbitrary lengths.
%%KF.3.1: The last sentence might be hard to follow. Fix this if have time.

We use a conditional variational encoder (CVAE)~\cite{sohn2015cvae} to capture the distribution of reachable goals $p(s' | s)$. In the CVAE, we define the decoder as $g(s, z)$ and the encoder as $q(z | s, s')$, where $z$ is the learned latent representation of the transitions and it is sampled from a prior probability $p(z)$. To propose a sequence of subgoals, we use $g(s, z)$ as the conditional subgoal generator. Conditioned on the initial state $s_0$, the first subgoal $\hat{s}_1$ can be generated as $\hat{s}_1 = g(s_0, z_1)$ given the sampled $z_1$. Then the $i_\text{th}$ subgoal can be recursively generated by sampling $z_i \sim p(z)$ and computing $\hat{s}_i = g(\hat{s}_{i - 1}, z_i)$ given the previous subgoal $\hat{s}_{i - 1}$. In this way, we could sample a sequence of i.i.d. latent representations $z_1, ..., z_K$ and recursively generate $\hat{s}_1, ..., \hat{s}_K$ conditioned on the initial state $s_0$ using the conditional subgoal generator.

% Training of the affordance model
    % - Trained for what
    % - Sample sequences for training
    % - Training with recursive generation. No backpropagation through time
    % - Handle compounding errors: Randomly choose between the gt context and the previously generated context.
    % - Training at different time scales
The CVAE is trained to minimize the evidence lower bound (ELBO)~\cite{kingma2014vae} of $p(s' | s)$ given the offline dataset $\mathcal{D}$. During training, we sample transitions $(s_t, s_{\tau})$ from the offline dataset to form the minibatches, where $\tau = t + \Delta t$ is a future step that is $\Delta t$ steps ahead. Instead of using a fixed $\Delta t$, we sample $\Delta t$ from a range for each transition to provide richer data. To encourage the trained model to be robust to compounding errors, we sample sequences composed of multiple states and use the subgoal reconstructed at the previous step as the context in the next step. Therefore, the objective for training the conditional subgoal generator is:
\begin{equation}
    \mathbb{E}_{q(z | s_t, s_{\tau})}||s_{\tau} - g(s_t, z)||^2 + \ensuremath{D_{KL}[q(z | s_t, s_{\tau}) || p(z)]}
    \label{eqn:elbo}
\end{equation}
where $\ensuremath{D_{KL}[\cdot || \cdot]}$ indicates the KL-Divergence.

% We train the conditional goal generator at different time scales.  

% Architecture of the affordance model
    % - Considerations: High efficiency & low compounding errors & diversity % samplable
    % - CCVAE
    % - UNet architecture 
    % - Discretization reduces the compounding errors 
% Latent state using VQVAE
    % - Use VQVAE encoding to generate states of high-dimensional.
    % - Pretraining VQVAE.
    % - Why this is not enough.
    % - Challenges. (static v.s. dynamic properties)
% To enhance the quality of the generated states, we use a U-Net architecture in the CVAE to preserve the information of different spatial granularities and perform vector quantization at the output layer. It is hard to conduct planning and checking if a goal is reached in the pixel space. Following the practice of \cite{}, we use a extract latent representations of the states using a Vector Quantized Variational Autoencoder (VQ-VAE)~\cite{}. For simplicity, we directly use $s$ to denote the VQ-VAE representations of the states in the rest of the paper. The VQ-VAE is pretrained on the prior data and its weights are fixed in the rest of the training process. 
% This paragraph should probably be moved to the Experiments section as an implementation detail.

% ----------------------------------------------------------------------------------------
\subsection{Efficient Planning in the Latent Space}

% ----------------------------------------------------------------------------------------
\begin{algorithm}[t]
\caption{$Plan(s_0, s_g, L, K, M, N)$}
\begin{algorithmic}[1]
\Require the initial state $s_0$, the goal state $s_g$, number of subgoals $K$, number of levels $L$, multiplier $M$, number of samples $N$.

\State Sample $N$ latent action sequences $\{z_{1:K}^i\}_{i=1}^N$.
\State Recursively generate subgoals $\{\hat{s}_{1:K}^i\}_{i=1}^N$ using $g(s, z)$.
\State Select $z_{1:K}^*$ and $\hat{s}_{1:K}^*$ of the lowest cost.
\State Update $z_{1:K}^*$ and $\hat{s}_{1:K}^*$ using MPPI.

\If{L = 1}
    \State \Return $\hat{s}_{1:K}^*$
\Else
    \State Denote $\hat{s}_0^* \leftarrow s_0$.
    \State Initialize the plan $\hat{\mathcal{S}}$ as an empty list
    \For{$i = 1, ..., K$}
        \State Append $Plan(\hat{s}_{i-1}^*, \hat{s}_{i}^*, L - 1, M, M, N)$ to $\mathcal{S}$
    \EndFor
    \State \Return $\hat{\mathcal{S}}$
\EndIf

\end{algorithmic}
\label{algo:planning}
\end{algorithm}
% \vspace{-5mm}

% To tackle the large variety of possible subgoal sequences, we build a planner that efficiently searches for sequences of subgoals in the latent space as shown in Algorithm~\ref{algo:planning}. Built upon the Model Predictive Path Integral (MPPI)~\cite{Gandhi2021RobustMP}, the planner  performs importance sampling by iteratively optimizing and perturbing the plan in the learned latent space of the conditional subgoal generator. The learned latent space captures the diverse interactions that the robot can perform given a state. A small perturbation in the latent space might result in very different generated subgoals. To address these issues, we propose two techniques to enable the planner to robustly search for the optimal plan in such a high-dimensional space.

%%SL.2.24: Generally, this paragraph is pretty hard to follow, and is a bit too vague to be well understood.

We build a planner that efficiently searches for sequences of subgoals in the latent space as shown in Algorithm~\ref{algo:planning}. To tackle the large search space of candidate subgoal sequences, we design a hierarchical planning algorithm that searches for subgoals in a coarse-to-fine manner and re-use the previously selected subgoals as candidates in new episodes.

% Hierarchical planning.
    % - Divide and conquer
    % - From coarse to fine
    % - Fixed resolution
    % - Similar to GCP, but conditional generation enables temporal extension and better generalization
    % - 
% We design a hierarchical planning algorithm with the conditional subgoal generator to reduce the search space. 
% Instead of generating the subgoals sequentially and evaluate the cost of the whole sequence holistically, we propose to generate and search for the subgoals in a coarse-to-fine manner. 
The hierarchical planning is conducted at $L$ levels with different temporal resolutions $\Delta t_1$, ..., $\Delta t_L$. The temporal resolution of each level is an integral multiple of that of the previous level, \ie, $\Delta t_i = M \Delta t_{i - 1}$, where $M$ is a scaling factor and is set to 2 in our experiments. We first plan for the subgoals $\hat{s}_1^1, \hat{s}_2^1, ...$ on the first level. Then the subgoals $\hat{s}_{1:K}^l$ of finer temporal resolution are planned on each level $l$ to connect the subgoals planned on the previous level $l-1$. Specifically, given the adjacent subgoals $\hat{s}_i^{l-1}$ and $\hat{s}_{i+1}^{l-1}$ produced on the previous level, we plan for a segment of  $M$ subgoals $\hat{s}_{i * M + 1}^{l}, ..., \hat{s}_{(i+1) * M}^{l}$ on the level $l$, by treating $\hat{s}_i^{l-1}$ and $\hat{s}_{i+1}^{l-1}$ as the initial state and final goal state in Eqn.~\ref{eqn:cost_function}. The planned segments are returned to the previous level and concatenated as a more fine-grained plan. For this purpose, we train $L$ conditional subgoal generators to propose subgoals that are $\Delta t_1$, ..., $\Delta t_L$ steps away, respectively. In contrast to the prior work~\cite{Pertsch2020LongHorizonVP}, the conditional subgoal generators enable us to plan for unseen goals that are beyond the temporal horizon of the demonstrations in the offline dataset by exploiting the compositional structure of the demonstrations. By recursively generating the subgoals across time at each level, we only need to enforce that the temporal resolution of the top level $\Delta t_L$ is smaller than since the the conditional subgoal generator $f^{1}(s, z)$ needs to be trained on trajectories at least $\Delta t_L + 1$ steps in length. 

% Latent plan buffer.
    % - Sensitive to the initialization
    % - In practice, we find that..., if we sample from the prior
    % - We keep a buffer
    % - Every episode, store
    % - Sample from both the prior and the buffer
We maintain a latent plan buffer for each level to further facilitate the planning with the conditional subgoal generator. After each episode, the selected latent representations on each level are appended to the corresponding latent plan buffer. In each target task, the subgoals are supposed to have the same semantic meaning. In spite of the variations of the initial and goals state in each episode, the optimal plans in the latent space can often be similar to each other. Therefore, we sample half of the latent representations from the prior distribution $p(z)$ and the other half from the latent plan buffer among the initial samples to enhance the chance of finding a close initial guess. 

We build our planner upon the model predictive path integral (MPPI)~\cite{Gandhi2021RobustMP}, which iteratively optimizes the plan through importance sampling. In each interaction, we perturb the chosen plan in the latent space with a small Gaussian noise as new candidates. 

\begin{figure}[t!]
    \centering
    \includegraphics[width=.48\textwidth]{ptp/figures/target_tasks.pdf}
    \vspace{-3mm}
    \caption{\textbf{Target tasks.} Three multi-stage tasks are designed for our experiments in the simulation and the real world respectively. In each target task, the robot needs to strategically interacts with the environment (\eg first takes out an object in the drawer then closes the drawer). The initial state and the desired goal state are shown for each task. }
    \vspace{-5mm}
    \label{fig:target_tasks}
\end{figure}

\begin{figure*}[t!]
    \centering
    \includegraphics[width=0.95\textwidth]{ptp/figures/sim_quantitative.pdf}
    \vspace{-3mm}
    \caption{
    \textbf{Quantitative comparison in simulation.} The average success rate across 3 runs is shown with the shaded region indicating the standard deviation. The negative x-axis indicates the epochs of offline pre-training and positive x-axis indicates epochs of online fine-tuning.  
    % This figure will show the success rates of all methods (including the baselines and ablations) for the three target tasks.
    Using offline learning and planning, our method PTP is able to solve these tasks partially (at 0 epochs).
    Then with online finetuning the performance improves further.
    In contrast, prior methods have lower offline performance and do not fine-tune successfully in most cases, as they do not collect coherent online data.
    }
    \vspace{-5mm}
    \label{fig:sim_quantitative}
\end{figure*}

% ----------------------------------------------------------------------------------------
\subsection{Cost Function For Feasible Subgoals}

% Planing objective. 
    % - Intuition: Reach the goal & transitions should be probable
    % - Reach the goal
    % - Sampling from the affordance model encourages the transitions to be likely.
% At beginning of each episode, we plan for a sequence of feasible subgoals that leads to the desired goal state of the target task from the initial state. The subgoals are chosen through sampling-based planning, in which we first sample $n$ sequences of subgoals $\hat{s}_{0:k}$ using the conditional goal generator and then evaluate the cost $c(\hat{s}_{0:k})$ of each sequence. 

To provide informative guidance to the policy $\pi(a | s, s_g)$, we would like that the final goal $s_g$ can be reached at the end of the episode while encouraging the transition between each pair of subgoals to be feasible within a limited time budget. As explained in Sec.~\ref{sec:preliminaries}, the goal state is considered to be reached when the Euclidean distance
%%SL.2.24: The Euclidean distance bit will come across as pretty confusing, given that previously we talked about using images -- so readers will wonder, does this mean that we are doing Euclidean distance over images? But that doesn't really make sense.
%%KF.3.1: Clarified.
between the last subgoal in the plan and the desired goal is less than a threshold $\delta$ in the learned latent space. The feasibility of each transition between adjacent subgoals can be measured using the goal-conditioned value function $V(s, s')$ trained by the reinforcement learning algorithm. Therefore, finding the subgoals $\hat{s}_{1:K}^*$ can be formulated as a constrained optimization problem: 
\begin{eqnarray}\label{eqn:constrained_optimization}
    \text{minimize} 
    & \quad & ||s_g - \hat{s}_K|| \\
    \text{subject to} 
    % & \quad & V(s_0, \hat{s}_1) \geq \delta\\
    & \quad & V(\hat{s}_i, \hat{s}_{i+1}) \geq \delta, \text{for $i = 0, ..., K-1$} \nonumber
\end{eqnarray}
where we use $\hat{s}_0 = s_0$ to denote the initial state for convenience. By re-writing Eq.~\ref{eqn:constrained_optimization} as a Lagrangian,
%%SL.2.24: The previous equation has three terms, but perhaps it can be written more concisely so that there is just one constraint (i.e., the line V(s_0, \hat{s}_1) \geq \delta is omitted), if we are careful with our notation.
%%KF.3.1: Fixed.
we obtain the cost function with a weight $\eta$:
\begin{equation}
    c(s_0, \hat{s}_{1:K}, s_g) = ||s_g - \hat{s}_K|| + \eta \sum_{i=0}^{K-1} V(\hat{s}_i, \hat{s}_{i+1})
    \label{eqn:cost_function}
\end{equation}
% where $\eta$ is a weight that balances the two terms. 
The details of our method are explained in Sec.~\ref{sec:implementation_details}.
\section{Experiments}

In our experiments, we aim to answer the following questions: 1) Can PTP propose and select feasible subgoals as plans for real-world robotic manipulation tasks? 2) Can the subgoals planned by PTP facilitate online fine-tuning of the goal-conditioned policies to solve target tasks unseen in the offline dataset? 3) How does each design option affect the performance of PTP?
% To answer these questions, we conduct experiments in both the simulation and the real world. 
Videos of our experimental results are available on the project website: \href{https://sites.google.com/view/planning-to-practice}{sites.google.com/view/planning-to-practice}

% \begin{enumerate}
%     \item Can PTP achieve strong performance after finetuning on goal-conditioned robotic manipulation tasks?
%     % \item Does PTP outperform prior methods on goal-conditioned simulated tasks? %%AVN.2.27 How should we separate sim vs real
    
%     \item Can PTP produce novel multi-stage strategies by composing the primitive behaviors seen in the prior data in new ways?
%     %%SL.2.24: This is a really important question, but maybe we can make this more explicit (i.e., by composing the primitive behaviors seen in the prior data in new ways)
%     %%PY.2.27: fixed
    
%     %%\item Can PTP generate feasible plans in latent space for solving long-horizon robotic manipulation tasks? %%AVN.2.27 What is the difference between these first two questions?
%     %%PY.2.27 Not much of a difference. Commenting out the second one.

% \end{enumerate}
% %%SL.2.24: maybe the finetuning question should be the first one, so as to avoid creating the impression that the finetuning is an "extra" thing and the method is mostly an imitation learning method -- basically, make it really clear that the online training is essential, and after showing that it works, we'll study all the different cool things that can be done with it (like composing skills in new ways via the planner)
% %%PY.2.27: fixed

% In the following sections, we study these questions in both simulated and real-world robotic manipulation domains. We evaluate and compare PTP with other model-based planning methods model-free RL methods that directly using final goals.
%%SL.2.24: it's not actual clear what its "model-free counterpart" refers to...
%%PY.2.27: fixed

\subsection{Experimental Setup}
\label{sec:experimental_setup}

% We construct a simulated platform to evaluate multi-step manipulation tasks using a real-time physics simulator [16]. As shown in Fig. 1, the workspace setup includes a 7-DoF Sawyer robot arm, a table surface, and a depth sensor (Kinect2) installed overhead. Up to 5 objects are randomly drawn from a subset of the YCB Dataset [17] and placed on the table. The Sawyer robot holds a short stick as the tool to interact with the objects to complete a specified task goal.

\textbf{Environment.}
As shown in Fig.~\ref{fig:target_tasks}, our experiments are conducted in a table-top manipulation environment with a Sawyer robot. At the beginning of each episode, a fixed drawer and two movable objects are randomly placed on the table. The robot can change the state of the environment by opening/closing the drawer, sliding the objects, and picking and placing objects into different destinations, \etc. At each time step, the robot receives a 48 x 48 RGB image via a Logitech C920 camera as the observation and takes a 5-dimensional continuous action to change the gripper status through position control. The action dictates the change of the coordinates along the three axes, the change of the rotation, and the status of the fingers. We use PyBullet~\cite{coumans2021} for our simulated experiments.  

% Our simulated and real-world experiments are identical in setup. The robot has 5 degrees of control: 3 dimensions of end-effector velocity control in Euclidean space, one dimension for rotation along the yaw axis of the end-effector, and one dimension to open and close the end-effector. For our simulation experiments, we use PyBullet \cite{coumans2021}. For our real-world experiments, we use a Sawyer robot.
%%SL.2.24: good place to reference a figure
%%KF.3.1: Fixed.

\textbf{Prior data.}
The prior data consists of varied demonstrations for different primitive tasks. In each demonstrated trajectory, we randomly initialize the environment and perform primitive interactions such as opening the drawer and poking the object. These trajectories are collected using teleoperation in the real world, and a scripted policy that uses privileged information of the environment (e.g., the object pose and the status of the drawer) in simulation. The trajectories vary in length from 5 to 150 time steps, with 2,344 trajectories in the real world and 4,000 in simulation.

% Randomly selected
% %%SL.2.24: you said before it was expert? now it's random?
% %%AVN.2.27 fixed
% rollouts from our prior data in our simulated multi-task environment are shown in Figure \ref{fig:sim_primitive_tasks}. The prior data is used to train the VQ-VAE and CVAE and used for offline RL pretraining. Each trajectory of our prior dataset consists of a noisy expert policy interacting with one of the three objects in our environment: a drawer, a slidable object, and a graspable object. This includes
% %%SL.2.24: this is easy to misunderstand as having a discrete hand-coded set of primitives (whereas in reality we have goals)
% %%PY.2.27: fixed
% \begin{itemize}
%     \item Opening or closing a drawer by the handle
%     \item Sliding a large object to an adjacent quadrant
%     \item Moving a small object via grasping in between inside the drawer, on top of the drawer, and any valid location on the table surface
% \end{itemize}

% One important note is that the notion of skills is not used in our model since our model is only conditioned on the image of the current state and the image of the goal state.
% % One important note is that we define these skills only for the purpose of collecting demonstrations. The notion of skills is not used in our model, and each skill can be broken down into more fine-grained behaviors. At the finest level, our planner aims to compose these primitive behaviors.

% %%SL.2.24: This is an important point, but I wonder if we can just tweak the phrasing in the preceding paragraphs to avoid referring to these as "skills" or "primitives" entirely or, failing that, find some way to clarify this paragraph?
% %%PY.2.27: Fixed. I tweaked the phrasing to avoid referring to these as "skills" or "primitives".

% The noisy expert policy used during data collection is a scripted policy with added Gaussian noise in simulation and a human teleoperator in the real-world. Every $K$ trajectories of data collection, we randomly select a new valid position and orientation for the drawer and objects. Further details about data collection are shown in Appendix \ref{appendix:experimental_details}.
% %%SL.2.24: I think readers will have *a lot* of questions about what this "noisy expert" is. The performance of the method and the difficulty of the tasks is very strongly dependent on understanding what the data is and where it comes from, and the description here is quite terse -- I think we should give more of an explanation, perhaps with some examples
% %%PY.2.27: fixed

\textbf{Target tasks.} 
In each target task, a desired goal state is specified by a 48 x 48 RGB image (same dimension with the observation). The robot is tasked to reach the goal state by interacting with the objects on the table. Task success for our evaluation is determined based on the object positions at the end of each episode (this metric is not used for learning). As shown in Fig.~\ref{fig:target_tasks}, we design three target tasks that require multi-stage interactions with the environment to complete. These target tasks are designed with temporal dependencies between stages (\eg the robot needs to first move away a can that blocks the drawer before opening the drawer). The transitions from the initial state to the goal state are unseen in the offline data. The episode length is 400 steps in simulation and 125 steps in the real world, which are much longer than the time horizon of the demonstrations. 

\textbf{Baselines and ablations.} 
We compare PTP with 3 baselines and 3 ablations. \textbf{Model-Free} uses a policy directly conditioned on the final goal and conducts online fine-tuning without using any subgoals. \textbf{LEAP}~\cite{Nasiriany2019PlanningWG} learns a variational auto-encoder (VAE)~\cite{kingma2014vae} to capture the prior distribution of states and plans for subgoals without conditioning on any context. \textbf{GCP}~\cite{Pertsch2020LongHorizonVP} learns a goal predictor that hierarchically generates intermediate subgoals between the initial state and the goal state. To analyze the design options in PTP, we also compare with variations of our method by removing the latent plan buffer (\textbf{PTP (w/o B)}), the hierarchical planning algorithm (\textbf{PTP (w/o H)}), and both of these two designs (\textbf{PTP (w/o H and B)}). All methods use the same neural network architecture in the goal-conditioned policy and are pre-trained on the same offline dataset.

\subsection{Implementation Details}
\label{sec:implementation_details}
% \kuan{In progress. Moving all implementation details here.}

%%In PTP, we collect an offline dataset $\mathcal{D}$, train representation learning, train offline RL, and finally run online RL for a specific environment. 



Following \cite{Khazatsky2021WhatCI}, we use a vector quantized variational autoencoder (VQ-VAE)~\cite{Oord2017NeuralDR} as the state encoder, which encodes a $48 \times 48 \times 3$ image to a 720-dimensional encoding. The conditional subgoal generator is implemented with a U-Net architecture~\cite{Ronneberger2015UNetCN} and decodes the subgoal from a 8-dimensional latent representations conditioned on the encoding of the current state. In our planner, we use $L = 3$, $K = 8$, $M = 2$, $N = 1024$, and we run MPPI for 5 iteration on each level. $g$ is trained to predict subgoals that are 15, 30, and 60 steps away. Implicit Q-Learning (IQL)~\cite{kostrikov2021iql} is used as the underlying RL algorithm for offline pre-training and online fine-tuning with default hyperparameters. We use the same network architectures for the policy and the value functions from ~\cite{Khazatsky2021WhatCI} for simulation experiments. For real-world experiments, we use a convolutional neural network instead. We use Adam optimizer with a learning rate of $3 \cdot 10^{-4}$ and a batch size of 1024. During training, we relabel the goal with future hindsight experience replay~\cite{Andrychowicz2017HindsightER} with $70\%$ probability. We use $\epsilon=3$ for the reward function defined in Sec.~\ref{sec:preliminaries}, $\eta = 0.01$ in Eqn.~\ref{eqn:cost_function}. 

% We also use a CVAE in $h$-space to generate subgoals. The VQ-VAE and CVAE are pre-trained on the offline dataset and its weights are fixed during online fine-tuning. We run offline RL using IQL ~\cite{kostrikov2021iql} and fine-tune in a specific environment with online RL. Our policy, Q-network, and value network are fully connected networks. Our reward function is $r(h, h_g) = -\mathds{1}_{\|h-h_g\| > \epsilon}$. For any transition $(s_t, a_t, r_t, s_{t+1}, s_g)$ during training, we also sample a new goal $s_g'$ with some probability, recompute $r_t'(s_{t+1}, s_g)$, and use this relabeled transition for off-policy Bellman updates. The relabeled goal $s_g'$ may come from future states in a transition~\cite{Andrychowicz2017HindsightER}, a random sample from the replay buffer~\cite{pong2018tdm}, or a random sample from learned goal distribution~\cite{nair2018rig}. 

% During training, we relabel the goal with future hindsight experience replay with $60\%$ probability and the next observation with $10\%$ probability. Our RGB images are 3x48x48, which the VQ-VAE encodes into length-720 vectors. We generate 8 subgoals when planning during finetuning, and the last subgoal is replaced by the final goal. For RL training, we use the Adam optimizer, a learning rate of $3 \cdot 10^{-4}$, weight decay of 0, batch size of 1024, reward scaling of 1, quantile of $.9$, $\beta=.01$, and $\epsilon=3$ (for sim) and $\epsilon=2$ (for real). During finetuning, we add a fixed Gaussian noise with standard deviation of $.15$. During evaluation, there is no noise added. We have a separate replay buffer for online data and offline data, and during training, sample $60\%$ from the online replay buffer and $40\%$ from the offline replay buffer. Our policy has 4 intermediate hidden layers of size 256, and our Q and V-networks have 2 intermdiate hidden layers of size 256. For our Q-network, we clamp the output at 0. Additional hyperparameters for training PTP and our code are available on the project website: \href{https://sites.google.com/view/planning-to-practice}{sites.google.com/view/planning-to-practice}.


% Prior work has shown that off-policy relabeling of goals can significantly improve the sample efficiency of goal-conditioned RL algorithms~\cite{Andrychowicz2017HindsightER}.
% %%SL.3.1: probably not the right place for prior work discussion... should keep this brief and factual if possible
% For any transition $(s_t, a_t, r_t, s_{t+1}, s_g)$, one can sample a new goal $s_g'$, recompute $r_t'(s_{t+1}, s_g)$, and use this relabeled transition for off-policy Bellman updates. The relabeled goal $s_g'$ may come from future states in a transition~\cite{Andrychowicz2017HindsightER}, a random sample from the replay buffer~\cite{pong2018tdm}, or a random sample from learned goal distribution~\cite{nair2018rig}.


% During finetuning and evaluation, we provide the robot an image of the goal state. To reach this goal state from its current state, the robot must sequence together multiple atomic behaviors. Importantly, the target tasks are temporally extended. The pair of initial state and the goal state in each target task is unseen during pretraining. The three target tasks are shown in Figure \ref{fig:sim_target_tasks}. Below are some brief descriptions of the three tasks:
% \begin{itemize}
%     \item Task 1. Close the drawer and then push the large object in front of the drawer.
%     \item Task 2. Push the large object out of the way and then open the drawer
%     \item Task 3. Take the small object out of the drawer, put it on top of the drawer, and then close the drawer.
% \end{itemize}

% Note that the atomic behaviors required to accomplish the target task are also temporally dependent. For instance, in task 1, the robot cannot push the large object in front of the drawer if it does not first close the drawer because the drawer is blocking the large objects' path. 
% We created three target tasks to fine-tune and evaluate over in both our simulation and real-world environment. 
% Specifically, we first arrange objects in the environment into an initial state. Then, we provide a goal image of objects in the environment arranged differently, so that the robot must accomplish a long-horizon task in order to reach a scene that resembles the goal image. Here, a long-horizon task consists of two temporally dependent primitive skills accomplished in sequence. 

%%SL.2.24: This is all rather vague, perhaps provide examples of start and goal states and explain how they require sequencing multiple atomic behaviors to reach the goals successfully
%%PY.2.27: fixed

%%KF.2.28: We should probably remove this figure. We can still put it on our website.
% \begin{figure}[t!]
%     \centering
%     \includegraphics[width=.45\textwidth]{figures/sim_primitive_tasks.png}
%     \caption{Prior data. Each row contains one randomly sampled trajectory from the prior data, which involves the scripted policy accomplishing a primitive skill.}
%     \label{fig:sim_primitive_tasks}
% \end{figure}


% \begin{figure}[t!]
%     \centering
%     \includegraphics[width=.4\textwidth]{example-image-a}
%     \caption{Real-World Tasks. This figure shows the initial and goal states in each target task in real world.}
%     \label{fig:real_target_tasks}
% \end{figure}


% \subsection{Fine-Tuning on Long-Horizon Tasks} 
% %%SL.2.24: the "non-episodic" bit is mentioned for the first time here, hard to understand
% %%AVN.2.27 changing the section title

% We evaluate our method and baselines on our three target tasks in both our simulation and real-world environment. The baselines we evaluate include

% \textbf{Model-Free (VAL)~\cite{Khazatsky2021WhatCI}} This method does not use planning with subgoals during exploration. Instead, the policy is conditioned on only the final goal.

% \textbf{GCP~\cite{Pertsch2020LongHorizonVP}} This method learns a goal-conditioned predictor which predicts an intermediate subgoal between the initial state and goal state that is fed into it. The predictor is trained hierarchically by recursively subdividing each part of the trajectory. The hierarchical planner used during exploration utilizes this predictor to optimize trajectories in a coarse-to-fine manner.

% \textbf{LEAP~\cite{Nasiriany2019PlanningWG}} This method learns a variational auto-encoder (VAE)~\cite{kingma2014vae} and plans over latent variables in the VAE latent space. LEAP used a temporal difference model (TDM)~\cite{pong2018tdm} to determine the reachability of latent subgoals. As there is no implementation of TDM that utilizes offline data, we instead use the same value function as our method to determine reachability, also making the two methods more comparable.

% % \textbf{SORB \cite{Eysenbach2019SearchOT}} This method samples one intermediate subgoal from the replay buffer during planning rather than from an affordance model.



\begin{figure}[t!]
    \centering
    \includegraphics[width=.48\textwidth]{ptp/figures/real_qualitative.pdf}
    % \vspace{-3mm}
    \caption{\textbf{Planned subgoal sequences.} Each row shows the sequence of subgoals produced by each method. The initial state and the final goal are shown at the two ends. 
    % Due to optimizing over an unconditional latent variable, subgoals from LEAP are incoherent. The sequence of images from GCP between the initial and goal image is smooth, but do not actually contain dynamically reachable subgoals. The planned subgoal sequence from our method in contrast successfully interpolates between the initial state and goal state to provide meaningful subgoals to the lower-level goal-conditioned policy.
    %%SL.3.1: same comment, please add a real caption
    }
    \vspace{-5mm}
    \label{fig:real_qualitative}
\end{figure}

\subsection{Quantitative Comparisons}

We evaluate PTP and baselines on three unseen target tasks. We use simulated versions of these tasks for comparisons and ablations, and real-world tasks, where all pretraining and finetuning uses only real-world data, to evaluate the practical effectiveness of the method.

\textbf{Simulation}. We first pre-train the goal-conditioned policy on the offline dataset for 100 epochs and the run online fine-tuning for the target task for 150 epochs. Each epoch takes 2,000 simulation steps (only during fine-tuning) and 2,000 training iterations. We run online fine-tuning using each method with 3 different random seeds. After each epoch, we test the policy in the target task for 5 episodes. We report the average success rate across 3 runs in Fig.~\ref{fig:sim_quantitative} where the negative x-axis indicates the offline pre-training epochs and positive x-axis indicates the online fine-tuning epochs.

As shown in Fig.~\ref{fig:sim_quantitative}, our full model consistently outperforms baselines with a large performance gap. The generated subgoals not only enables the pre-trained policy to achieve higher success rate by breaking down the hard problems into easier pieces, but also introduces larger performance improvements during online fine-tuning. After fine-tuning for 150 epochs, the policy achieves the success rates of 84.9\%, 59.9\%, 49.3\% in the three target tasks respectively. Compared to the policy pre-trained on the offline dataset, the performance is significantly improved (+31.6\%, +37.8\%, and +13.8\%). When directly using the final goal or subgoals generated by baseline methods, the policy's performance plateaus at around 0.0\% to 30.0\% and does not improve much during online fine-tuning. 

We found that the hierarchical planner and the latent plan buffer are crucial for PTP's performance. Without these two design options, the planner often suffers from the large search space of possible subgoal sequences and the resultant success rates decrease. The latent plan buffer significantly improves the performance of non-hierarchical PTP while it has a minor effect on hierarchical PTP.   

\textbf{Real-world evaluation.} We pre-train the policy for 200 epochs and fine-tune it for 10 epochs. In each epoch, we run 10,000 training iterations and collect 1,000 steps in the real world. We train on three target tasks which are shown in Figure~\ref{fig:target_tasks}, and report the success rate of the goal-conditioned policy before and after online fine-tuning in Table~\ref{fig:real_quantitative}. Planning enables the robot to succeed partially with just the offline initialized policy, achieving success rates of $12.5\%, 75.0\%$ and $25.0\%$ on the three tasks. (When the offline policy is conditioned on only the final goal image without planning, the success rate is $0\%$.) Then in each task, we fine-tune to a significantly higher success rate.

Qualitatively, at the beginning of fine-tuning, the robot often fails, deviating from the planned subgoals or colliding with the environment.
With the planned subgoals, the original long-horizon task is broken down to short snippets that are easier to complete.
Even if a subgoal is not reached successfully at first, the data is useful to collect additional experience and fine-tune the policy.
After fine-tuning for 4-5 epochs, we already observe that the robot's performance reaching subgoals during training time significantly improves, collecting even more coherent and useful data.
After 10 epochs, we achieve success rates of $62.5\%, 100.0\%$ and $50.0\%$.
In comparison, GCP cannot provide useful guidance to the policy when the generated goals are noisy.
%%SL.3.1: Given that this is our main result, it would be good to have a bit more in-depth discussion of what the results actually are, and maybe point to some qualitative results in a figure somewhere.


% \begin{figure}[t!]
% \captionof{table}{The real-world success rates before and after online fine-tuning. The tasks are described in Sec.~\ref{sec:experimental_setup}.}
\begin{table}[t!]
    \normalsize
    \centering
    \caption{The real-world success rates before and after online fine-tuning. The tasks are described in Sec.~\ref{sec:experimental_setup}.}
    \begin{tabular}{ c|c|c }
        \centering
        Task & \begin{tabular}{@{}c@{}}PTP (Ours) \\ Offline $\rightarrow$ \text{Online} \end{tabular} & \begin{tabular}{@{}c@{}} GCP \\ Offline $\rightarrow$ \text{Online} \end{tabular} \\
        \hline
        % (1) Close drawer, slide can & $25.0\% \rightarrow \textbf{62.5\%}$ & $0.0\% \rightarrow 0.0\%$ \\
        % (2) Slide can, open drawer & $12.5\% \rightarrow \textbf{62.5\%}$ & $0.0\% \rightarrow 0.0\%$ \\
        % (3) Poke object, close drawer & $12.5\% \rightarrow \textbf{100\%}$ & $12.5\% \rightarrow 37.5\%$ \\
        % Task A & $25.0\% \rightarrow \textbf{62.5\%}$ & $0.0\% \rightarrow 0.0\%$ \\
        % Task B & $12.5\% \rightarrow \textbf{62.5\%}$ & $0.0\% \rightarrow 0.0\%$ \\
        % Task C & $12.5\% \rightarrow \textbf{100\%}$ & $12.5\% \rightarrow 37.5\%$ \\
        Task A & $12.5\% \rightarrow \textbf{62.5\%}$ & $12.5\% \rightarrow 0.0\%$ \\
        Task B & $75.0\% \rightarrow \textbf{100.0\%}$ & $50.0\% \rightarrow 75.0\%$ \\
        Task C & $25.0\% \rightarrow \textbf{50.0\%}$ & $25.0\% \rightarrow 12.5\%$ \\
    \end{tabular}
    \label{fig:real_quantitative}
    \vspace{-6mm}
\end{table}
% \end{figure}

\subsection{Generated Subgoals}

In Fig.~\ref{fig:real_qualitative}, we present qualitative results of the generated subgoals for each task in the real world. Each row shows a sequence of generated subgoals produced by the planner in each method. In all the three target tasks, PTP successfully plans for a sequence of subgoals that can lead to the desired final goal. The transition between adjacent subgoals are feasible within a short period of time. By comparison, both of the baseline methods fail to generate reasonable plans. Without conditioning on the current state, LEAP~\cite{Nasiriany2019PlanningWG} can hardly produce any realistic images of the environment. Most of the generated subgoals are highly noisy images with duplicated robot arms and objects. The quality of the subgoals produced by GCP~\cite{Pertsch2020LongHorizonVP} is higher than that of LEAP but still much worse than ours. GCP cannot generalize well for the initial state and the goal state that are out of the distribution of the offline dataset, which contains only short snippets of demonstrations.
\section{Conclusion and Discussion}

We presented PTP, a method for real-world learning of temporally extended skills by utilizing planning and fine-tuning to stitch together skills from prior data.
First, planning is used to convert a long-horizon task into achievable subgoals for a lower level goal-conditioned policy trained from prior data.
Then, the goal-conditioned policy is further fine-tuned with active online interaction, mitigating the distribution shift between the offline data and actual states seen during rollouts.
This procedure allows robots to extend their capabilities autonomously, composing previously seen data into more complicated and useful skills.

%% Use plainnat to work nicely with natbib. 

\bibliographystyle{plainnat}
\bibliography{references}

%% \newpage
% \clearpage
% \appendix
% \input{appendix-a-environment-details}
% \input{appendix-b-implementation-details}

\end{document}