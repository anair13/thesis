Thus far, we have presented \METHOD assuming that we have access to a goal-reaching policy, allowing us to separately analyze how we can maximize $\HG$.
However, in practice we do not have access to such a policy, and this section discusses how we concurrently train a goal-reaching policy.

Maximizing $I(\SF; \G)$ can be done by simultaneously performing \METHOD and training a goal-conditioned policy to minimize $\HGS$, or, equivalently, maximize $-\HGS$.
Maximizing $-\HGS$ requires computing the density $\log p(\G \mid \SF)$, which may be difficult to compute without strong modeling assumptions.
However, for any distribution $q$, the following lower bound on $-\HGS$:
\begin{align}\nonumber
-\HGS
    &= \E_{(\G, \SF) \sim q}\left[
        \log q(\G \mid \SF)
    \right]
+ \kld{p}{q}
\\\nonumber
&
    \geq \E_{(\G, \SF) \sim q}\left[
        \log q(\G \mid \SF)
    \right],
\end{align}
where $\KL$ denotes Kullback–Leibler divergence as discussed by \citet{barber2004information}.
Thus, to minimize $\HGS$, we train a policy to maximize the reward
\begin{align}\nonumber
    r(\SF, \G) = \log q(\G \mid \SF).
\end{align}

The RL algorithm we use is reinforcement learning with imagined goals (RIG)~\citep{nair2018rig}, though in principle any goal-conditioned method could be used.
RIG is an efficient off-policy goal-conditioned method that solves vision-based RL problems in a learned latent space.
In particular, RIG fits a $\beta$-VAE~\citep{higgins2016beta} and uses it to encode observations and goals into a latent space, which it uses as the state representation.
RIG also uses the $\beta$-VAE to compute rewards, $\log q(\G \mid \SF)$.
Unlike RIG, we use the goal distribution from \METHOD to sample goals for exploration and for relabeling goals during training~\citep{andrychowicz2017her}.
Since RIG already trains a generative model over states, we reuse this $\beta$-VAE for the generative model $\pg$ of \METHOD.
To make the most use of the data, we train $\pg$ on all visited state rather than only the terminal states, which we found to work well in practice.
To prevent the estimated state likelihoods from collapsing to zero, we model the posterior of the $\beta$-VAE as a multivariate Gaussian distribution with a fixed variance and only learn the mean.
We summarize RIG and provide details for how we combine \METHOD and RIG in \autoref{sec:rig-and-full-method} and describe how we estimate the likelihoods given the $\beta$-VAE in \autoref{sec:likelihood-estimation-vae}.
